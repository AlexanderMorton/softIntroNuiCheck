

%The Standard Model top quark is a fermion that interacts as described
%in the Chapter~\ref{chp:theory}. It
%was discovered at Fermilab
%in 1995~\cite{PhysRevLett.74.2626}~\cite{PhysRevLett.74.2632}.


This study focuses on the semileptonic decays of the top-antitop ($t\bar{t}$) system
and aims at measuring its cross section as a function of the number of jets produced in the final state.
%What is actually measured in this study is the number of events generated for each jet multiplicity configuration, that is the cross section multiplied by the
%integrated luminosity.
%The conversion to a cross section means simply dividing the result by the integrated luminosity used for this measurement, $L = 4.7 \ifb \pm 0.2 \ifb$.

The motivation behind this study is discussed in the next section, followed by details about the samples used for the analysis signal and background estimate.
The unfolding method used, which corrects for detector effects is presented with a discussion of the systematic effect that it produces.
The analysis has been made public as a conference note by ATLAS~\cite{ttjets_confnote}.

\section{Motivation}

As mentioned in Chapter~\ref{chp:theory},
the top quark has a very large mass compared to the other quarks
and, therefore, it is expected that it has a very large coupling to the Standard Model Higgs boson, which is central to the Electroweak Symmetry Breaking mechanism.
Furthermore, the top quark has a very short lifetime and decays before hadronising.
In its decay, the top quark is expected to radiate gluons, generating extra jets in the final state. This extra radiation
would appear as higher order diagrams, in addition to the lowest order decay and can be predicted using Quantum Chromodynamics.
A measurement of the production of this extra radiation is important to test how well the Standard Model describes the production of both the top quark and the associated QCD radiation.

Furthermore, the top-antitop production is a main background in many physics analyses,
such as the $t\bar{t}H$ search and in searches of supersymmetric models.
A good understanding of \ttbar production is essential to have a good modelling of other analyses' backgrounds.

In the current analysis, it is proposed to measure the production cross section of the \ttbar system as one of the top quarks decays leptonically and the other top quark
decays hadronically. The measurement of the production cross section is done as a function of the number of jets in the final state, which are
related to the amount of radiation produced in the \ttbar decay. The analysis is, therefore, an important test of the Standard Model description of the top quark.
The main elements of the analysis include the signal selection, which improves the signal-to-background ratio, the background estimation
and the unfolding procedure (described in Section~\ref{sec:ttjets_unfolding}), which corrects for the detector effects.

In the next section an account of the signal modelling through different simulation
methods used in the analysis is given.
The signal modelling is important to estimate the detector effects which need to be corrected in the unfolding procedure (Section~\ref{sec:ttjets_unfolding}).
The background is simulated or estimated through data-driven methods and it is also discussed in the next sections.
The event selection is discussed next, which attempts to mitigate the effect of the background.
The unfolding procedure is detailed then, followed by the method used for uncertainty propagation.
Finally the resulting data to simulation comparison before and after the unfolding is shown.

\section{Top-antitop signal simulation and background estimates}
\label{sec:ttjets_background}

%Although the measurement of the cross section will be done in data, the detector affects the results, changing the measured observable.
%The detector contribution to the results should be inverted to correct for its influence and this is achieved through the
%unfolding procedure.
The signal is simulated as it
is understood in the Standard Model to measure the effect of the detector. It is also simulated using different generators
and configurations to test the agreement with data after the effect of the detector is corrected for.
All sources of background are estimated as well, so that this contribution can be subtracted from data.

As a first step, the signal must be
defined in terms of its final state objects, so that the selection can be properly justified.
The signal contains an electron, muon or tauon and a neutrino as a result of the $W$-boson decay, two $b$-quarks
from the top decays and two other quarks from the other $W$-boson decay.
The $\tau$ lepton decays either leptonically into an electron or muon and neutrinos,
or it decays hadronically, generating jets. If the $\tau$
lepton decays hadronically, it will generate jets, which we do not intend to select and it will be considered as part of the background.

Therefore, the detector measures an electron or muon, missing transverse energy due to the undetected neutrino
and at least four jets, since more jets
could be produced as a result of QCD radiation.
As there is either an electron or a muon in the final state, this splits the signal into two distinct channels to be taken into consideration,
which are referred to henceforth
as the ``electron channel'' and the ``muon channel''.
In this chapter, for what follows, the word ``lepton'' will be used to refer only to the final state electron or muon, with the possible $\tau$
leptonic decays being indirectly included as electron or muon decays.


The $t\bar{t}$ signal is simulated with different fixed order calculations and matching schemes for the parton shower.
The Alpgen v2.13 generator~\cite{alpgen} 
with the CTEQ6L1~\cite{cteq6} PDF set is used as the main reference sample to derive correction factors in the unfolding procedure.
This sample was chosen since it predicts the jet multiplicity distribution well at reconstruction-level. A systematic
uncertainty associated with the choice of this sample is calculated at a later stage.

The Alpgen reference sample is generated with zero up to four exclusive and five inclusive additional partons
produced as extra radiation. Herwig v6.520~\cite{herwig1,herwig2}
is used for the parton showering and the fragmentation and the MLM~\cite{mlm_matching} parton-jet matching scheme is applied\footnote{
With parameters \texttt{ETCLUS 20 GeV, RCLUS 0.7, ETACLUS 6.0}.} to avoid double counting between the matrix element calculation and the parton shower.
MC@NLO~\cite{mcatnlo_gen}\footnote{Using the CT10~\cite{ct10} PDF set and interfaced with Herwig for the parton shower.} and
Powheg~\cite{powheg}\footnote{Using the CTEQ 6.6~\cite{cteq6} PDF set and interfaced with Pythia~\cite{pythia} for the parton shower,
using the AUET2B-CTEQ 6L1 tune.} generators are used when comparing the final unfolded result in Section~\ref{sec:ttjets_unfolded}.
MC@NLO is also used to estimate the systematic effect caused by
the unfolding procedure. Jimmy~\cite{jimmy} is used with each sample produced with Herwig, for the underlying event simulation with the AUET1 tune.
An Alpgen v2.14 sample is also produced using Pythia~\cite{pythia} for the parton shower and the CTEQ5L PDF set to test
the systematic effect of the parton shower.
Furthermore, to test the effect of the ISR/FSR models, Alpgen samples are generated with different renormalisation scales associated
with doubling and halving the scale $Q$ at which $\alpha_S(Q)$ is calculated in the matrix element, while keeping the $\alpha_S$ configuration for the parton shower and
the PDF set fixed. The nominal $\alpha_S$ value
is set to be $0.118$ at the $Z$
boson mass scale.
The Alpgen+Pythia nominal and $\alpha_S$ variation samples are produced using the Pythia Perugia 2011 tune~\cite{perugia}. The top mass was set to be
$m_{t\bar{t}} = 172.5$ GeV and its production cross section was normalised to $\sigma_{t\bar{t}} = 167^{+17}_{-18}$ pb in all samples used\footnote{Obtained from approximate NNLO QCD~\cite{hathor}
calculations
at the $m_{t\bar{t}}$ point used.}.

Many background events can be selected as they have a similar final state configuration as the signal, either because the final state particles are the same,
or because there are experimental effects which lead the selection mechanism to tag the background as the signal,
such as misidentification of jets as leptons.
The selection procedure in this
analysis emphasizes the \ttbar signal over the background, rejecting significantly more background than signal, but it is not enough
to completely remove the background.
It is necessary to estimate the remaining background in the data after the selection has been made,
and subtract them.

% Explain W+jets data-driven method and Z+jets
The main background to the $t\bar{t}$ production estimate in the semileptonic channel is W+jets production, which could
be falsely identified as being part of the signal if the $W$ boson decays leptonically and there are extra jets produced by radiation.
The shape of the W+jets kinematic distributions is estimated from Alpgen v2.13, using the CTEQ6L1 PDF set and interfaced with Herwig for the parton shower,
but the normalisation of the distribution and the
heavy flavour content is not well
predicted from the simulation and a partially data-driven method is used. The W+jets estimate also includes an estimate of the heavy and light flavour content
partially calculated from data, which is important to understand the effect of the $b$-tagging cut on the W+jets distribution. This method is explained in more details
in Section~\ref{sec:ttjets_wjets}. It is worth noting that, since the samples of $W + b\bar{b} +$ jets, $W + c\bar{c} +$ jets, $W + c +$ jets and $W +$ light-jets are produced
separately, a heavy flavour overlap removal procedure is used to remove the overlap between the heavy flavour content of the samples, using a $\Delta R$ match between
the simulation-level Anti-$k_t$ jets and the reconstruction-level Anti-$k_t$ jets and removing the event if they represent an overlap between the samples.

% Explain QCD data-driven method
Due to the high number of jets involved in the analysis, the QCD multi-jet production
is very low. Therefore,
the QCD multi-jet production simulation cannot be used to generate enough events and other methods are used to estimate it.
The main source of misidentification of QCD multi-jet events as \ttbar decay products happens if one of the
jets is misindentified as a lepton.
This motivated a data-driven estimate of this background's contribution, which is done by estimating how often one of
the jets in a QCD enriched data sample is mistaken for a lepton. This method was used in this analysis and details on how it was done are mentioned in
Section~\ref{sec:ttjets_qcd}.

% Explain single top, Z+jets and diboson backgrounds simulation
The Z+jets sample is simulated using Alpgen v2.13 with the CTEQ 6L1 PDF set, using Herwig for the parton shower. An angular matching is used, as in the $W+$ jets case to
remove the overlap between the $Z+$jets and the $Z+b\bar{b}+$jets samples.
The $t$-channel single top quark sample was generated using the AcerMC~\cite{acermc} generator, while the $Wt$- and $s$-channel predictions were simulated
using MC@NLO. Each single top sample was normalised to the approximate NNLO calculation of each respective
channel~\cite{stop1,stop2,stop3}\footnote{$\sigma_{t, \textrm{t-channel}} = 64.5^{+2.6}_{-1.7}$ pb, $\sigma_{t, \textrm{s-channel}} = 4.6^{+0.2}_{-0.2}$ pb,
$\sigma_{t, \textrm{Wt-channel}} = 15.7^{1.2}_{-1.2}$ pb.}. The $WW$, $ZZ$ and $WZ$ production are generated using Herwig.

\section{Top-antitop event selection}
\label{sec:ttjets_selection}

%The objective of the selection procedure is to select the signal with minimal background, maximising the signal-to-background ratio.
%The remaining background is subtracted from the data.
%The signal is also affected, but it is expected that the selection is such that the background
%is much more reduced than the signal.

In what follows, the requirements used to select events are described.
The cuts used to select reconstruction-level events are organised in a set of main items:
a trigger-related selection, lepton selection, jet selection and missing energy requirements.
%The last subsection is dedicated to the fiducial region definition at particle-level, which restricts the measurement
%performed to a phase space volume.

\subsection{Trigger and pile up-related selection}

A first step in the selection procedure is to use the trigger to select events tagged as containing at least one electron with $p_T > 20\gev$ or $p_T > 22\gev$
depending on the relevant data taking period,
or at least one muon with $p_T > 18\gev$,
according to the selection channel\footnote{\texttt{EF\_e20\_medium} for periods B-J, \texttt{EF\_e22\_loose} for period K and
\texttt{EF\_e22vh\_medium1} or \texttt{EF\_e45\_medium1} for periods L to M, in the electron channel; \texttt{EF\_mu18} for periods B to I and \texttt{EF\_mu18\_medium} for periods J to M, in the muon channel.}.
A second step demands that at least four tracks were used to reconstruct the position of the primary vertex, which
works as a quality cut on the event reconstruction and it reduces the effect of multiple proton-proton interactions.

%Another quality cut is applied to remove
%the event if it contains anti-$k_t$ jets with $R=0.4$~\cite{antiktalgo}, $p_T > 20$ GeV, $E > 0$ GeV and if it has been tagged as a badly reconstructed jet.

\subsection{Lepton selection}

As a next step, it is important to demand that one and only one well identified and isolated
electron or muon was detected in the electron or muon channel, respectively.
A few quality cuts must be required for the lepton, to remove misreconstructed leptons or non-prompt leptons, which come from the
decay products of other
particles and not from the top quark's $W$ boson decay. The electrons are required to have a transverse momentum of at least $25$ GeV and it must have a well
reconstructed track that matched a calorimeter energy deposition region. Electrons are also required to be in the pseudorapidity range
$|\eta| \in [0, 1.37) \cup (1.52, 2.47]$ to exclude the ``crack'' region of the calorimeters, in which the energy of the electron is not well
reconstructed. For muons, the Inner Detector track is 
required to match the Muon Spectrometer track and it must be in the
pseudorapidity range of $|\eta| < 2.5$.
Furthermore, the sum of energy in the calorimeters in a region of $\Delta R < 0.2$ around the muon, excluding its own energy is required to be less than $4$ GeV
and the sum of the track momenta in a region of $\Delta R < 0.3$ (excluding the muon's track momentum) is also required to be less than $4$ GeV.
For the electron the same energy sum and track momenta sum is calculated, but the cut is adjusted for each particle, so that the cut has a flat 90\% efficiency
in the electron's transverse momentum~\footnote{In the ATLAS jargon, the variables calculated are called \texttt{Etcone20} and \texttt{Ptcone30}. In
the muon channel, it is demanded that $\mathtt{Etcone20} < 4\gev$ and $\mathtt{Ptcone30} < 4\gev$, while for the electron channel the cuts are adjusted so that
an efficiency of 90\% is kept as a function of transverse momentum.}.

An important quality cut to be taken into account is to reject electrons that overlap with jets if $0.2 < \Delta R (e, \textrm{jet}) < 0.4$ and remove jets that satisfy
$\Delta R (e, \textrm{jet}) < 0.2$. The latter removal step attempts to reduce the effect of leptons which were misidentified as jets,
while the former electron removal step is necessary because the electron identification and reconstruction corrections to be applied
at a later stage assume well separated electrons and jets. A similar overlap removal quality cut is applied to muons, which are required to be far away from jets, according
to the criteria $\Delta R(\mu, \textrm{jet}) > 0.4$, removing non-prompt muons that might be produced from the semileptonic $b$-quark decay and that would be close to the
$b$-jet.

It is also important to reject the event if a lepton of a different type is found in the event, that is, in the electron channel, no good quality muons
should exist and in the muon channel, no good quality electrons should exist,
which removes the effect of the dilepton top-antitop decays and $Z$ boson decays into two leptons. To increase the purity of the
selected sample, if there is at least one lepton of the other type with a transverse momentum greater than $15\gev$, the event is rejected~\footnote{Note that
this is smaller than the $25\gev$ requirement for the selected lepton.}.
The selected lepton is also required
to match the lepton selected
by the trigger system in the corresponding electron or muon trigger selection~\footnote{It is required to match one of the selected trigger electrons (for the electron channel)
or muons (for the muon channel),
since another object might also have been accepted by the trigger system as an electron or muon.
This would happen because the trigger system might misidentify one or more of the objects. No demand is made on the multiplicity of the selected trigger electrons or muons.}. 

\subsection{Jet selection}

Although the total number of jets from the $t\bar{t}$ decay channel of interest includes at least four jets, some of the jets
might not be detected within the fiducial volume of the detector. The unfolding procedure could benefit from an estimate
of these jets being lost, by including the events with three jets as well, so that migrations of events from the four-jets category to the three-jets
category could be corrected.
Therefore, the event is required to contain at least 3 jets within $|\eta| < 2.5$ and a jet vertex fraction (see Section~\ref{sec:atlas_jet} for the definition
of the jet vertex fraction) greater than $0.75$ to reduce the
pile up contribution. The transverse momentum cut on each jet is chosen to be $25\gev$, $40\gev$, $60\gev$ and $80\gev$ in four versions
of this analysis, that is, the analysis is repeated four times, each one demanding that all selected jets have a specific minimum energy.
This
procedure shows the effect of jets at different transverse momenta ranges in the jet multiplicity distribution.
The particular choice
of these values is arbitrary, though.

\subsection{Missing energy requirements}

To reject backgrounds without neutrinos in the final state, such as the fully hadronic decay of the top-antitop
pair and the QCD multi-jets background, the total missing transverse energy in the event is required to be greater than $30\gev$.

The transverse mass between the lepton and the missing transverse energy, $\met$, is defined as:

\begin{equation}
\displaystyle
m_T = \sqrt{2 p_T^\ell \met (1-\cos \alpha)},
\label{eq:ttjets_mtw}
\end{equation}
where $p_T^\ell$ is the lepton's transverse momentum, $\met$ is the missing transverse energy, and $\alpha$ is the
azimuthal angle between $p_T^\ell$ and $\met$. This variable has an end-point at the mother particle's mass, which, for a top-antitop decay,
should be the $W$-boson mass.
Taking advantage of this, a cut is applied on $m_T$, which is required to be greater than $35\gev$.
Finally, at least one of the jets, with $p_T > 25$ GeV is required to be $b$-tagged. The $b$-tagging criteria in the analysis was chosen based on
the 70\% efficiency operating point of the Neural Network based tagger MV1 (see Section~\ref{sec:atlas_btag} for more information on $b$-tagging).
One further $b$-jet could have been required,
since in principle $t\bar{t}$ events should contain at least two $b$-jets, but due to the low
efficiency of the $b$-tagging algorithm\footnote{Notice that increasing the efficiency point would also increase the false-identification rate, reducing
the purity of the selected sample.}, making extra demands on the $b$-tagging criteria would increase the statistical uncertainty significantly.

%\subsection{Fiducial region definition}
%
%Besides optimising the signal-to-background ratio, the event selection restricts the measurement to a fiducial region, on which the detector
%would operate optimally. These selection requirements are applied identically in the reconstruction-level and particle-level simulated
%leptons, jets and missing transverse energy and are not corrected by the unfolding procedure, restricting the resulting measurement
%to a phase space volume. The unfolding method, therefore, corrects the effects of the detector up until the following restrictions.
%
%The fiducial region is defined by the minimum transverse momentum requirement on the leptons of $25\gev$,
%including the requirement
%that no other lepton is detected with a transverse momentum of at least $15\gev$.
%The pseudo-rapidity requirements on the leptons of $|\eta| \in [0, 1.37) \cup (1.52, 2.47]$ for electrons and $|\eta| < 2.5$ for muons
%are also required. The jets' pseudo rapidity is restricted to $|\eta| < 2.5$ and their minimum momentum is set to be $25\gev$,
%$40\gev$, $60\gev$ and $80\gev$ in four different versions
%of the analysis. The missing transverse energy and transverse mass requirements are set to be the same as the reconstruction-level
%ones, namely, $\met > 30\gev$ and $m_T > 35 \gev$.
%The overlap removal requirements are applied in the same way as at reconstruction-level.
%An isolation requirement is also applied at the particle-level selection, which will be detailed in the unfolding section.
%At least three jets, and one lepton are required, as in the reconstruction-level selection.
%
%These fiducial
%requirements are applied at the signal simulation (using Alpgen+Herwig) to estimate the correction factors for the unfolding procedure,
%which restricts the resulting measurement to the corresponding phase space volume.

\section{Data-driven W+jets background estimate}
\label{sec:ttjets_wjets}

As mentioned previously, the W+jets production rate and flavour fractions are not well described in simulation and a partially data-driven method is used in this work.
This method was applied in this and other top quark related analyses and the estimate of the factors mentioned in this section were performed by other ATLAS working
groups and not the author of this
document himself. The method is briefly introduced for clarity and completeness only.

The data-driven W+jets estimate is performed in three stages. In the first step, the flavour fraction of the W+jets background is constrained before the $b$-tagging
requirement.
A control region is defined 
with the same selection cuts, but constraining the jet multiplicity in the final state to be one or two jets, instead of at least 
three jets. The control region described is enriched in the W+jets background and the small contribution from other backgrounds
and \ttbar is subtracted. The backgrounds and \ttbar contributions' are estimated using Monte Carlo simulation in
the Control Region. This results in
a final event count for the 1- and 2-jet channels, before and after the $b$-tagging selection, which will be referred to as $W^{\textrm{data}}_{i, \textrm{pre-tag}}$ and
$W^{\textrm{data}}_{i, \textrm{tagged}}$, for $i = 1,2$.

Equations can be written for $W^{\textrm{data}}_{1, \textrm{pre-tag}}$, $W^{\textrm{data}}_{1, \textrm{tagged}}$, $W^{\textrm{data}}_{2, \textrm{pre-tag}}$ and $W^{\textrm{data}}_{2, \textrm{tagged}}$ as a function
of eight independent flavour fractions as unknowns, which represent the fractions of $W+b\bar{b}+$ jets, $W+c\bar{c}+$ jets, $W+c+$ jets and
$W+$ light-jets in the 1- and 2-jet bins
before $b$-tagging. The $b$-tagging probabilities estimated in simulation are used to express the tagged quantities as a function of the untagged flavour fractions in the
relevant equations. The ratio between the 1- and 2-jet bin event counts are calculated from simulation to relate the values in the set of the equations. Furthermore,
the ratio of number of events coming from $W+c\bar{c}+$ jets and $W+b\bar{b}+$ jets is estimated in simulation to reduce even more the number of degrees of freedom in the
equations. With these constraints, the three equations remain with only three degrees of freedom, which can be extracted, by solving the linear system.
This procedure results in an estimate for the flavour fractions before $b$-tagging
for the $W+b\bar{b}+$ jets (which are related to the $W+c\bar{c}+$ jets by the ratio estimated in simulation), the $W+c+$ jets,
and the $W+$ light-jets. These fractions are applied to the relevant simulation data samples to improve the flavour fraction components from this data-driven measurement,
but no normalisation change is made at this stage.

The second step in this estimate is to use a data-driven method to calculate the normalisation of the W+jets background after the selection but without the $b$-tagging
requirement.
It can be done by noting that the production rate of $W^{+}+$jets is bigger than that of $W^{-}+$jets, since there are more up valence quarks in the protons than down valence
quarks and that the ratio between the production rates of $W^{+}+$jets and $W^{-}+$jets, $r_{MC}$, is more precisely calculated from simulation than the rates themselves.
The estimate is done by counting the number of data events in the signal region which produce positively charged leptons and negatively charged leptons, that is, the
events that come
respectively, from the $W^{+}$ and $W^{-}$ decays.
In addition,
other processes involved in this measurement produce, to good approximation, equal number of positively and negatively charged leptons and, since we are only interested
in the asymmetry between these rates, they would not interfere in this estimate. The remaining background is subtracted.
The number of W+jets events before the $b$-tagging requirement can, then, be calculated using Equation~\ref{eq:wjets}, in which $r_{MC}$ is the ratio between $W^{+}$ and
$W^{-}$ events produced in simulation, $D_{W^{+}}$ and $D_{W^{-}}$ are the amount of detected $W^{+}$ and $W^{-}$ in data and $W_{\textrm{pre-tag}}$ represents the total
number of W+jets events produced (before the $b$-tagging requirement).

\begin{equation}
\displaystyle
W_{\textrm{pre-tag}} = N_{W^{+}} + N_{W^{-}} = \left(\frac{r_{MC} + 1}{r_{MC} - 1}\right)  ( D_{W^{+}} - D_{W^{-}} )
\label{eq:wjets}
\end{equation}

%After the previous steps, the flavour fractions and the normalisation of the W+jets background are fixed before the $b$-tagging requirement.
Each W+jets flavour component sample is simulated after the selection including the $b$-tagging requirement, with each data sample weighted by the estimated flavour fraction.
The final step estimates the
effect of the $b$-tagging requirement in simulation and applies it to the previously calculated normalisation\cite{toppairxsec7tev,ttjets_confnote}.
%The total normalisation of W+jets events after the
%$b$-tagging requirement is then calculated through the Equation~\ref{eq:wjets_btag}, in which $f_{2,\textrm{tagged}}$ represents the fraction of events in the 2-jet
%bin which passed the $b$-tagging criteria and $k_{2 \rightarrow 4}$ represents the fraction of $b$-tagged events found in the 4-jet bin over the $b$-tagged events found in the
%2-jet bin\cite{toppairxsec7tev}.

%\begin{equation}
%\displaystyle
%W_{\textrm{tagged}} = W_{\textrm{pre-tag}} \times f_{2,\textrm{tagged}} \times k_{2 \rightarrow 4}.
%\label{eq:wjets_btag}
%\end{equation}

\section{Data-driven QCD multi-jets background estimate}
\label{sec:ttjets_qcd}

The calculation of the data-driven QCD multi-jets estimate was not done by the author in this analysis\footnote{Although it is done by the author
in a latter analysis,
the $t\bar{t}$ resonances search at $\sqrt{s} = 8$ TeV. See Chapter~\ref{chp:ttbarres8}.}, but the procedure is mentioned here for completeness, as it was for the case for the W+jets estimate
in the previous section.
The QCD multi-jets background is estimated by a method called Matrix Method, which associates weights to data events with looser lepton
identification requirements to generate distributions for this background.

Events for this QCD estimate are categorised as containing a ``loose'' lepton, which only satisfy a
looser lepton identification criteria, with no isolation requirement, or they are categorised as containing a ``tight'' lepton,
which also satisfies the
standard selection lepton identification criteria.
The real data estimate and the Monte Carlo simulation estimates in the analysis contains only events that pass the
event selection using the ``tight'' lepton criteria. The QCD estimate described below contains real data events that pass the selection using the ``tight'' and ``loose'' lepton definitions. Each event in the QCD estimate
is weighted depending on two variables introduced below. These variables,
called $\epsilon_\textrm{eff}$ and $\epsilon_\textrm{fake}$,
are estimated in a real lepton-enriched region and a QCD-enriched region.

A QCD-enriched control region is defined orthogonal to the selection region given by
the requirements for the analysis (Section~\ref{sec:ttjets_selection}).
The control region must be enriched in the QCD multi-jets
background, so that the effect of the $t\bar{t}$ signal and the other backgrounds is reduced.
For this analysis, the control region was defined by inverting the missing transverse energy cut and tightening it to $E_T^{\textrm{miss}} < 20$ GeV (in both
electron and muon channels) and by inverting the
transverse mass cut (in the electron channel and in one of the muon channel estimates).
An alternate estimate in the muon channel defines
the control region by requiring events which have a muon with large impact parameter with respect to the primary vertex. In spite of the difference of control region definitions,
both results in the muon channel follow a similar procedure, which is described below.

The main point of the method is to estimate the fraction of events that satisfy the ``tight'' lepton criteria amongst all events
that satisfy the ``loose'' lepton criteria in the QCD events using the control region, and in true lepton (that is, non-QCD multi-jets) events,
using a region enriched in events containing one real lepton.
Monte Carlo simulations
might be used for the non-QCD region, matching the simulation meta-data to the reconstruction-level leptons to increase the purity.
A Tag and Probe method in $Z \rightarrow \ell \ell$ data events can also define a lepton-enriched region, using the $Z$ mass window as a selection criteria
and using the two leptons to measure the probability of detecting a ``tight'' lepton as a probe, given that the tag lepton satisfies the ``loose'' criteria. The latter
method is used in this analysis. The two procedures are equivalent, since (see Section~\ref{sec:ttjets_corrections}) a Tag And Probe method is used to calculate
the lepton selection efficiency in data and use it to correct simulation. In this way, the corrected lepton in simulation has the same selection efficiency as real data.

The fractions determined in the QCD-enriched control region and in the lepton-enriched region can be carried on to the \ttbar
selection region and used to define weights for each event, depending on whether
it passes the ``loose''-only or ``tight'' criteria.

The number of ``loose'' events, $N_{\mathrm{loose}}$, includes a component, $N^{\mathrm{fake}}_{\mathrm{loose}}$, coming from QCD multi-jets
events and a component, $N^{\mathrm{real}}_{\mathrm{loose}}$, coming from mis-identified signal events. And similarly for the number of events satisfying the ``tight''
criteria, as in Equation~\ref{eq:ttjets_qcd_def}.

\begin{eqnarray}
\displaystyle
N_{\mathrm{loose}}&=&N^{\mathrm{fake}}_{\mathrm{loose}} + N^{\mathrm{real}}_{\mathrm{loose}}\\
N_{\mathrm{tight}}&=&N^{\mathrm{fake}}_{\mathrm{tight}} + N^{\mathrm{real}}_{\mathrm{tight}}
\label{eq:ttjets_qcd_def}
\end{eqnarray}

In this method, one wishes to estimate $N^{\mathrm{fake}}_{\mathrm{tight}}$ in real data, using the selection procedure for the current
analysis.
We define

\begin{equation}
\displaystyle
\epsilon_{\mathrm{eff}} = \frac{N^{\mathrm{real}}_{\mathrm{tight}}}{N^{\mathrm{real}}_{\mathrm{loose}}}
\label{eq:ttjets_qcd_eff}
\end{equation}
 and

\begin{equation}
\displaystyle
\epsilon_{\mathrm{fake}} = \frac{N^{\mathrm{fake}}_{\mathrm{tight}}}{N^{\mathrm{fake}}_{\mathrm{loose}}},
\label{eq:ttjets_qcd_fake}
\end{equation}
which can be used in Equation~\ref{eq:ttjets_qcd_def} to obtain

\begin{equation}
\displaystyle
\begin{pmatrix}
N_{\mathrm{loose}} \\
N_{\mathrm{tight}}
\end{pmatrix}
=
\begin{pmatrix}
1 & 1 \\
\epsilon_{\mathrm{fake}} & \epsilon_{\mathrm{eff}}
\end{pmatrix}
\times
\begin{pmatrix}
N^{\mathrm{fake}}_{\mathrm{loose}} \\
N^{\mathrm{real}}_{\mathrm{loose}}
\end{pmatrix}.
\label{eq:ttjets_qcd_form}
\end{equation}

Note that the number of events that satisfy the tight and loose selections $N_{\mathrm{loose}}$ and $N_{\mathrm{tight}}$
in the left-hand side can be found out
in data by applying the event selection in data using the ``tight''~\footnote{The one previously mentioned in the
event selection for this analysis.} or the slightly-altered ``loose'' definition for leptons. If the $\epsilon_{\mathrm{fake}}$ and
$\epsilon_{\mathrm{eff}}$ parameters are known, the unknown values of $N^{\mathrm{fake}}_{\mathrm{loose}}$ and $N^{\mathrm{real}}_{\mathrm{loose}}$
could be discovered. Furthermore, these quantities are easily related to the ``tight'' ones ($N^{\mathrm{fake}}_{\mathrm{tight}}$ and
$N^{\mathrm{real}}_{\mathrm{tight}}$) by Equations~\ref{eq:ttjets_qcd_eff} and~\ref{eq:ttjets_qcd_fake}. If we want to calculate
$N^{\mathrm{fake}}_{\mathrm{tight}}$, which represents the number of QCD events that pass the analysis lepton-definition,
in the analysis' selection region, one can simply calculate the number of events in data
that pass the ``tight'' and ``loose'' selections and invert the last equation, which results in:

\begin{equation}
\displaystyle
\begin{pmatrix}
N^{\mathrm{fake}}_{\mathrm{tight}} \\
N^{\mathrm{real}}_{\mathrm{tight}}
\end{pmatrix}
=
\frac{1}{\epsilon_{\mathrm{eff}} - \epsilon_{\mathrm{fake}}}
\begin{pmatrix}
\epsilon_{\mathrm{fake}} \epsilon_{\mathrm{eff}} & - \epsilon_{\mathrm{fake}} \\
- \epsilon_{\mathrm{fake}} \epsilon_{\mathrm{eff}} & \epsilon_{\mathrm{eff}}
\end{pmatrix}
\times
\begin{pmatrix}
N_{\mathrm{loose}} \\
N_{\mathrm{tight}}
\end{pmatrix}.
\label{eq:ttjets_qcd_invform}
\end{equation}

An interesting element from the last equations is that they are linear on the number of events selected.
This implies, that one can divide the data samples
in separate sets and $N^{\mathrm{fake}}_{\mathrm{tight}}$ can be calculated in each one, with its total value for the combined
region being the sum of the results in each subset.
This feature can be used to apply this equation in an event-by-event basis, creating one separate set for each event.
With this setting, each event either passes only the ``loose'' selection or the event passes both the ``loose'' and ``tight'' selections,
which would correspond to the $\begin{pmatrix} N_{\mathrm{loose}}\\ N_{\mathrm{tight}}\end{pmatrix} = \begin{pmatrix} 1 \\ 0\end{pmatrix}$
or the $\begin{pmatrix} N_{\mathrm{loose}}\\ N_{\mathrm{tight}}\end{pmatrix} = \begin{pmatrix} 1 \\ 1\end{pmatrix}$ configurations,
respectively. The resulting $N^{\mathrm{fake}}_{\mathrm{tight}}$ value can then be interpreted as a per-event weight to be applied in
real data, depending on whether it passes the ``tight'' selection or only the ``loose'' selection.

If the data event passes the ``tight'' and ``loose'' selections, it is weighted by:

%From this result, it is possible to apply weights on events that satisfy the ``loose''-only or ``loose''+``tight'' criteria to obtain the total number
%of events that pass the ``tight'' criteria in the \ttbar selection region and which represent only the QCD multi-jets background contribution,
%$N^{\mathrm{fake}}_{\mathrm{tight}}$. Indeed, from the coefficient matrix in Equation~\ref{eq:ttjets_qcd_invform}, the events that pass the ``tight'' %and ``loose'' criteria
%should be weighted by the sum of the coefficients of $N_{\mathrm{loose}}$ and $N_{\mathrm{tight}}$ in the
%first row:

\begin{equation}
\displaystyle
w_{\mathrm{tight}} = \frac{\epsilon_{\mathrm{fake}}}{\epsilon_{\mathrm{eff}} - \epsilon_{\mathrm{fake}}} \times ( \epsilon_{\mathrm{eff}} - 1 ),
\label{eq:mmqcd_wtight}
\end{equation}
and, events that satisfy only the ``loose'' criteria, should be weighted by:

\begin{equation}
\displaystyle
w_{\mathrm{loose}} = \frac{1}{\epsilon_{\mathrm{eff}} - \epsilon_{\mathrm{fake}}} \times ( \epsilon_{\mathrm{fake}} \epsilon_{\mathrm{eff}} ).
\label{eq:mmqcd_wloose}
\end{equation}

The only information necessary for the weights are $\epsilon_{\mathrm{eff}}$ and $\epsilon_{\mathrm{fake}}$, which are estimated
in the QCD-enriched and lepton-enriched regions defined previously. The weighting itself is applied on events that pass the
selection used in the analysis.

Since the weights are only calculated for each event, the kinematic information of the event can be exploited to parametrise
$\epsilon_{\mathrm{eff}}$ and $\epsilon_{\mathrm{fake}}$ as a function of a few variables on which they show a large dependency.
These variables might vary in different conditions and this parametrisation can improve the shape of the resulting QCD estimate.

These weights are calculated for each event
in data, using the standard analysis selection (defined
in the Section~\ref{sec:ttjets_selection}). The leptons in the selection
are also allowed to pass only a looser criteria (as defined in the current
section) if they fail the standard selection with the tighter (default)
criteria. Depending on whether the event satisfies the standard selection
with the loose or tight lepton criteria, the event is categorised as
``loose'' or ``tight'' and the appropriate weight (Equation~\ref{eq:mmqcd_wloose} or~\ref{eq:mmqcd_wtight}, respectively) is associated to it.
The histogram for the desired observable (in the current analysis,
it would be the jet multiplicity) is filled with the
associated event weight and the result is the QCD estimate for such
observable for events passing only the tight selection.

Note that
in this process, events that are not accepted, if the standard lepton
definition is used but are accepted for a looser lepton definition in data,
are also included in the QCD estimate, weighted appropriately.
According to the equations above, it is expected that the weighted sum
of the events satisfying the tight lepton definition and the looser
lepton definition is an estimate of the number of QCD multi-jet events
that would pass only the tight lepton definition.


\section{Corrections applied in simulation}
\label{sec:ttjets_corrections}

A set of corrections are applied in simulation to obtain a better agreement between the simulation and data.
Some of the corrections are implemented as reweighting procedures, in which the final histograms calculated are weighted based on some characteristics of the event
so that some of the selection efficiencies which are different in data and simulation can be taken into account, or that other differences between data and simulation
are reduced, if the simulation setup could not be made exactly as the data configuration.
Other correction procedures use statistical pseudo-experiments, which change the kinematics in Monte Carlo.

In some Monte Carlo simulation samples, the events should receive a positive or negative weight according to the MC@NLO procedure described in~\cite{mcatnlo}. Although this
involves reweighting, this is not a correction on itself, but a necessary procedure related to how the event generator works.
Each sample is also reweighted so that it corresponds to the correct cross section. This is done by normalising a simulation result dividing it by
the total number of events generated and multiplying it by the process' production cross section and the data luminosity.

Amongst the reweighting corrections, one of them refers to matching the average number of multiple interactions per bunch crossing $<\mu>$ to data.
This is implemented by measuring the average number of interactions per bunch crossing in data, $<\mu>_{\textrm{data}}$, and building a normalised histogram
with this distribution: $P_{\mu,\textrm{data}}$. A similar distribution is calculated with the average number of simulated interactions for Monte Carlo samples,
$P_{\mu,\textrm{MC}}$. With these distributions, a simulated event with an avarage number of interactions $<\mu>_{\textrm{MC}}$ receives a weight of
$P_{\mu,\textrm{data}}(<\mu>_{\textrm{MC}})/P_{\mu,\textrm{MC}}(<\mu>_{\textrm{MC}})$, so that the simulated and reweighted $<\mu>$ distributions would match the estimated
distribution in data. The $<\mu>$ distribution in data, however, can have different shapes in different running periods and this histogram was calculated separately for different data-taking periods.
%B until D, E until H, I, J, K and L until M.
To account for these different $\mu$ distributions in data, the Monte Carlo simulation events
are separated into subsets with a number of events proportional to the fraction of luminosity in each data period mentioned.
Each subset of the Monte Carlo with a fraction of events proportional to a data period range $X$ is
weighted by $P_{\mu,\textrm{data period X}}(<\mu>_{\textrm{MC}})/P_{\mu,\textrm{MC}}(<\mu>_{\textrm{MC}})$, where $P_{\mu,\textrm{data period X}}$
represents the probability distribution of $<\mu>$ in the period range $X$.
This allows for a reweighting that takes into account features 
from each data collection period in the $<\mu>$ distribution shape. This procedure is used to
get a similar effect from multiple interactions coming from the simulation.

Due to the demand that at least one jet satisfies the $b$-tagging criteria,
the $b$-tagging efficiency and mistag rates could also be a source of discrepancy between data and the simulation, which is why the simulation is reweighted to
match the $b$-tagging performance in data. The simulation $b$-tagging efficiency and mistag rates are measured for different simulation flavours and different $\eta$ and $p_T$
ranges, while in data, different data-driven methods are used to estimate them as a function of $p_T$ and $\eta$ (see Section~\ref{sec:atlas_btag}).
The efficiencies are available for each jet and the
$b$-tagging requirement is applied for any jet in the event (that is, not to a specific jet), so an event-wide
weight is built by assuming that there is no correlation on the $b$-tagging performance effect between the different jet kinematics. This is implemented by multiplying
the scale factor for each jet in a single event weight. The scale factor for each $b$-tagged jet is calculated by the ratio of the efficiency of tagging the jet in
data and in Monte Carlo simulation. For jets that failed the $b$-tagging criteria, the scale factor is calculated as the ratio of the tagging inefficiencies
(one minus the efficiency for $b$-tagging a jet) in data and Monte Carlo simulation. Details on the scale factors calculated for 2011 ATLAS data are described in~\cite{btag2011}.

The ratio of efficiencies in data and Monte Carlo is also used to correct for the identification
requirements applied in the lepton selection. The document in~\cite{muonperf2010} describes the methods used for muon efficiency estimation in 2010 data, which
were used similarly for 2011 data. The electron performance is detailed in~\cite{electron2010} for 2010 data as well, but a similar procedure was used for 2011 data.
The jet vertex fraction requirement also has a selection efficiency slightly different in simulation and data and
the efficiency ratio is used as a scale factor for this correction.

Besides the weighting procedure to correct for efficiency differences in data and simulation, the difference in the resolution and energy scales must also be considered,
since the performance in simulation might be slightly different from that in real data collisions. A correction is applied in data for the electron energy scale, using a correction
factor measured from data in $Z$-boson and $J/\psi$ decays in electron pairs, and performance studies using $W$-boson decays into electrons and neutrinos.

The
electron energy resolution is corrected in Monte Carlo simulation by multiplying the electron cluster energy by a pseudo-random
sample of a Gaussian distribution with mean one and standard deviation given by $\sqrt{r_{\textrm{data}}^2 - r_{\textrm{MC}}^2}$, where $r_{\textrm{data}}$ is
the resolution in data and $r_{\textrm{MC}}$ is the resolution in Monte Carlo. The electron performance studies in ATLAS are described in~\cite{electron2010} for 2010
data, although the procedure is similar for the 2011 data.
The muon transverse momentum is also smeared to take into account the difference in the muon resolution
in data and Monte Carlo, using events in real data
with two muons in the final state to measure the resolution. Momentum scale corrections are also applied in muons, based on the shift of the $Z$-boson peak in measured
data. The procedure used to estimate the correction factors is explained in~\cite{muonres2010} using 2010 data, although the procedure is similar for the 2011 study.

The jet energy scale is also corrected for discrepancies in data and simulation, as described in~\cite{jes2011}. The jet energy scale is calculated from energy balance
in decays of $Z$-boson (which decays into electron or muon pairs) and one extra jet, so that the jet energy calibration can be expressed as a function of the lepton energies.
Decays with photons and jets are also used. See Chapter~\ref{chp:atlas} for more details on how these corrections are calculated.

\section{Data to signal and background comparison}

After the selection procedure described previously, the number of reconstructed jets can then be analysed,
as in Figure~\ref{fig:Alpgen_Njets}
and compared with
data, using the same event selection for all backgrounds.
Data-driven techniques are used to estimate the W+jets background
contribution and the QCD jets contribution.
Alpgen was used to simulate the \ttbar signal in these plots.
%The equivalent result with MC@NLO simulation for the signal is shown in
%Figures~\ref{fig:MCAtNLO_NjetsEl} and~\ref{fig:MCAtNLO_NjetsMu}.

It can be seen that the data and the estimated background agree within systematic uncertainties. This result
is going to be used as a first step for the unfolding procedure, detailed in the next sections, in which
it is important to subtract the background contribution from data, before moving to the unfolding of the
\ttbar  signal. The systematic variations on the signal and background were summed in these plots, but for the
background subtraction procedure, which is applied in data, only the background systematic uncertainties are taken into
account and propagated to the ``data - background'' estimate, which is detailed in Section~\ref{sec:ttjets_unfold_unc}.

\begin{sidewaysfigure}[tpbp]
\centering
\subfloat{\includegraphics[width=0.49\linewidth]{figures/ttbarjets/plotInGroupOnlyHists_groupSigBkgAlpgen_NjetsEl.eps}}
\subfloat{\includegraphics[width=0.49\linewidth]{figures/ttbarjets/plotInGroupOnlyHists_groupSigBkgAlpgen_NjetsMu.eps}}
\caption{Jet multiplicity in the electron (left) and muon (right) channels using Alpgen simulation for the $t\bar{t}$ signal ($p_T > 25$ GeV).}
\label{fig:Alpgen_Njets}
\end{sidewaysfigure}

More plots showing the kinematic properties of the selected events are available in Appendix~\ref{app:ttjets_control}.
The numeric yields for each of the signal and backgrounds is given in Table~\ref{tab:yield} for the events that pass the $25\gev$ transverse momentum threshold.
The systematic uncertainties shown include all variations in the reconstruction process, such as
the jet energy scale uncertainty, the $b$-tagging mistag rate and efficiency uncertainties, the
missing transverse energy uncertainties.
More details about the systematic uncertainties will be given in the next section.

\input{tables/YieldElAlpgen}
\input{tables/YieldMuAlpgen}

As the jet multiplicity shown depends strongly on the transverse momentum cut applied on the jets before counting them, further
comparisons were done with different cuts applied to the jets. Figure~\ref{fig:Alpgen_Njets40} shows
the results using a 40 GeV cut on the jets transverse momentum. Figure~\ref{fig:Alpgen_Njets60} uses a
60 GeV threshold and Figure~\ref{fig:Alpgen_Njets80} uses a 80 GeV threshold.
A comparison of the Alpgen+Herwig \ttbar simulation with different transverse momentum
requirements on the Anti-$k_t$ $R=0.4$ jets is shown in Figure~\ref{fig:Alpgen_JetCuts}.

% plots showing ttbar signal with different jet p_T cuts

\begin{sidewaysfigure}[tpbp]
\centering
\subfloat{\includegraphics[width=0.49\linewidth]{figures/ttbarjets/plotInGroupOnlyHists_groupSigBkgAlpgen_Njets40El.eps}}
\subfloat{\includegraphics[width=0.49\linewidth]{figures/ttbarjets/plotInGroupOnlyHists_groupSigBkgAlpgen_Njets40Mu.eps}}
\caption{Jet multiplicity in the electron (left) and muon (right) channels using Alpgen simulation for the $t\bar{t}$ signal ($p_T > 40$ GeV).}
\label{fig:Alpgen_Njets40}
\end{sidewaysfigure}

\begin{sidewaysfigure}[tpbp]
\centering
\subfloat{\includegraphics[width=0.49\linewidth]{figures/ttbarjets/plotInGroupOnlyHists_groupSigBkgAlpgen_Njets60El.eps}}
\subfloat{\includegraphics[width=0.49\linewidth]{figures/ttbarjets/plotInGroupOnlyHists_groupSigBkgAlpgen_Njets60Mu.eps}}
\caption{Jet multiplicity in the electron (left) and muon (right) channels using Alpgen simulation for the $t\bar{t}$ signal ($p_T > 60$ GeV).}
\label{fig:Alpgen_Njets60}
\end{sidewaysfigure}

\begin{sidewaysfigure}[tpbp]
\centering
\subfloat{\includegraphics[width=0.49\linewidth]{figures/ttbarjets/plotInGroupOnlyHists_groupSigBkgAlpgen_Njets80El.eps}}
\subfloat{\includegraphics[width=0.49\linewidth]{figures/ttbarjets/plotInGroupOnlyHists_groupSigBkgAlpgen_Njets80Mu.eps}}
\caption{Jet multiplicity in the electron (left) and muon (right) channel using Alpgen simulation for the $t\bar{t}$ signal ($p_T > 80$ GeV).}
\label{fig:Alpgen_Njets80}
\end{sidewaysfigure}

\begin{sidewaysfigure}[tpbp]
\centering
\subfloat{\includegraphics[width=0.49\linewidth]{unfoldingB/alpgen/JetCuts_cut80_AllEl.eps}}
\subfloat{\includegraphics[width=0.49\linewidth]{unfoldingB/alpgen/JetCuts_cut80_AllMu.eps}}
\caption{Jet multiplicity in the electron (left) and muon (right) channels using Alpgen+Herwig simulation for the \ttbar signal with different Anti-$k_t$ jet transverse momentum cuts applied. In this figure, for the $60\gev$ plot, the 7 jet bin represents events with $\geq 7$ jets, and in the $80\gev$ plot, the 6 jet bin represents events with $\geq 6$ jets.}
\label{fig:Alpgen_JetCuts}
\end{sidewaysfigure}


%The Alpgen samples use leading-order simulation, but they simulate the final
%states of the $t\bar{t}$ decay with up to 5 partons due to the extra radiation, while
%the MC@NLO simulation, despite including next-to-leading-order terms, simulates only the
%first few extra partons due to radiation. The consequence is that the MC@NLO simulation has a
%large disagreement at higher jet multiplicity bins.

%\begin{figure}[tpbp]
%\begin{center}
%\begin{minipage}[t]{0.48\linewidth}
%\includegraphics[width=\linewidth]{out_rel17_v2000/ttbarjets/plotInGroupOnlyHists_groupSigBkgMCAtNLO_NjetsEl.eps}
%\caption{Jet multiplicity in the electron channel using MC@NLO simulation for the $t\bar{t}$ signal.}
%\label{fig:MCAtNLO_NjetsEl}
%\end{minipage}
%\begin{minipage}[t]{0.48\linewidth}
%\includegraphics[width=\linewidth]{out_rel17_v2000/ttbarjets/plotInGroupOnlyHists_groupSigBkgMCAtNLO_NjetsMu.eps}
%\caption{Jet multiplicity in the electron channel using MC@NLO simulation for the $t\bar{t}$ signal.}
%\label{fig:MCAtNLO_NjetsMu}
%\end{minipage}
%\end{center}
%\end{figure}

\section{Systematic uncertainties estimate at reconstruction level}
\label{sec:ttjets_syst}

Before correcting for the detector effects, all systematic effects related to the
objects reconstruction which affect the selection or the determination of the jet multiplicity are evaluated,
by varying each of the parameters used in the calculations.
The full list of uncertainties is given in
Tables~\ref{tab:elyieldsyst} and~\ref{tab:muyieldsyst}. A description of the sources of systematic uncertainties is described here.

The jet energy scale is one of the main uncertainties, which has an effect that grows with the jet multiplicity. This can
be explained, by noting that, for a higher jet multiplicity, the event has more low transverse momentum jets, which are more
sensitive to the minimum transverse momentum requirement.
To estimate the uncertainty on the jet energy scale,
the jet's four-momentum is varied based on the uncertainties generated by the effect of close-by jets, the effect of multiple proton-proton interactions,
and the flavour composition of the jets (light quark versus gluon). For events with more than seven jets, the uncertainty with seven jets was used.
An additional $p_T$ dependent uncertainty was associated to jets that match B-hadrons.
The jet reconstruction efficiency was measured as the
fraction of jets reconstructed from tracks that match a calorimeter jet and the difference observed was taken as a jet reconstruction efficiency uncertainty~\cite{ttjets_jer},
which was applied in this measurement by randomly removing a fraction of the jets in the simulation events accordingly.

Jets were also smeared according to the jet energy resolution uncertainty,
after checking that there is an agreement in this quantity between data and simulation~\cite{ttjets_jer,jetreso2010}. No nominal correction was
applied for the jet energy resolution, since studies~\cite{jetreso2010}
show good agreement between data and simulation.


Another one of the main uncertainties, is the $b$-tagging performance, which was measured in data and simulation. The differences in the data and simulation $b$-tagging
efficiencies and mistag rates were corrected in the simulation by applying a scale factor to all events~\cite{mv1note}.
The uncertainty in the measurement of the $b$-tagging performance was propagated to the scale factors and its effect in the final observable was estimated by varying the
scale factors used in simulation accordingly. The effect of this uncertainty is significant in the low jet multiplicity bins, but it does not grow as much as the jet energy
scale uncertainty with the number of jets in the event. The scale factors used and their uncertainties can be seen in Section~\ref{sec:atlas_btag},
in Figure~\ref{fig:btag_ptrel}.

To mitigate the effect of multiple proton-proton interactions, the jets under consideration are required to have a jet vertex fraction greater than 0.75 in
absolute value~\cite{ttjets_confnote}.
A scale factor was applied based on the efficiency ratio in data and simulation events and the uncertainty in the efficiency measurement was propagated
to the scale factor, as mentioned in Section~\ref{sec:atlas_jet}.
The leptons' trigger, reconstruction and identification efficiencies were also measured in simulation and in data and scale factors were also derived with the appropriate
uncertainties (see Section~\ref{sec:atlas_electron} and Section~\ref{sec:atlas_muon}). The efficiencies were measured in data through $Z$ and $W$ boson decays.

The missing transverse energy is measured by summing all corrected lepton and jet energies and demanding that the transverse energy is conserved. Calorimeter cells
not associated to reconstructed objects with $p_T > 20$ GeV, have their energies added in the missing transverse energy ``CellOut'' component.
``Soft'' jets, that is, cells from jets with $p_T > 7$ GeV and $p_T < 20$ GeV, and the ``CellOut'' components are varied by 6.6\% to estimate the effect of the multiple
proton-proton interactions in the selection. This number was calculated by studying the dependency of the missing transverse energy on the multiple particle
interactions~\cite{ttjets_confnote}. Consult Section~\ref{sec:atlas_met} for more information.

The Alpgen \ttbar prediction has an uncertainty from the choice of the CTEQ6L1 Parton Distribution Function (PDF)~\cite{cteq6}, which was
evaluated by using the MSTW PDF set at leading-order with 68\% Confidence Level~\cite{mstw,pdflhc}
to reweight the \ttbar sample. The systematic uncertainty related to the PDF choice was calculated by
including the difference in the nominal value caused by the choice of the PDF, as well as adding in quadrature the difference between the nominal value and the
results when using all eigenvector sets from the MSTW PDF~\cite{pdflhc}. The uncertainty due to the parton shower modelling was also estimated by comparing the
results obtained with Alpgen+Herwig and Alpgen+Pythia \ttbar simulations. The ISR/FSR variations between the Alpgen+Pythia central yields and the
Alpgen+Pythia yields with $\alpha_S$ increased and
decreased were also used to estimate the extra radiation impact in the results.
The Powheg+Pythia and the Alpgen+Pythia \ttbar samples were compared to include the systematic uncertainty related to the difference
between fixed order matrix element calculations and associated matching schemes.

The W+jets charge asymmetry measurement also has an uncertainty associated to it, from statistical uncertainties on the data-driven measurement and uncertainties
from the simulation-dependent part of the method, including lepton and jets reconstruction, charge mis-identification, Monte Carlo generators, backgrounds and
Parton Distribution Function uncertainties. The heavy flavour fraction estimate includes, besides simulation uncertainties, a 25\% uncertainty when extrapolating the results
from the 2-jet bin to higher jet multiplicity. The W+jets Monte Carlo simulation also includes an uncertainty from the choice of the renormalisation and factorisation scales
(estimated by varying the \texttt{iqopt3} parameter~\footnote{This changes the method in which the scale is defined in Alpgen, by multiplying the standard scale by $\sqrt{m_W^2 + p_{T,W}^2}$. See~\cite{alpgen} for more information.} in Alpgen) and from generator cuts (estimated by varying the \texttt{ptjmin} parameter~\footnote{This changes the minimum transverse momentum cut for light jets in Alpgen. See~\cite{alpgen} for more information.} in Alpgen).

The integrated luminosity in data was measured using van der Meer~\cite{ttjets_lumi} scans and
it was used to normalise many simulation samples accordingly. Its uncertainty was found
to be 3.9\%\footnote{An uncertainty of 3.9\% was used instead of 3.7\%, as in the reference, due to the higher uncertainties in the second half of the 2011 data taking.}~\cite{ttjets_lumi}.
The single top production cross section uncertainties were taken to be 4\% in the $t$-channel, 4\% in the $s$-channel and 8\% in the $Wt$-channel from approximate NNLO calculations.
The diboson production cross section uncertainty was taken to be 5\%. For Z+jets, 4\% added in quadrature with 24\% per jet was taken for the theoretical
cross section uncertainty.

The QCD multi-jet background uncertainty can be estimated in the muon channel from the shape difference between the two methods used for the data-driven estimate.
In the electron channel, the missing transverse energy selection requirement was varied between 15 GeV and 25 GeV for the control region.
The normalisation uncertainty was taken to be 50\% in the electron channel and 20\% in the muon channel as a result of comparing the Matrix Method estimates
with other methods.

\input{tables/SystematicYieldElAlpgen}
\input{tables/SystematicYieldMuAlpgen}

\section{Unfolding the effect of the detector}
\label{sec:ttjets_unfolding}

When reconstructing the jet, lepton and missing transverse energy quantities,
the detector changes the physical observables in many ways.
An unfolding procedure is
necessary to measure the actual cross section as a function of the jet multiplicity and correct the effect of the
detector.
The reference used as the particle level result is obtained after applying only the transverse momentum, isolation and $\eta$ requirements
on the Monte Carlo simulation metadata.
The Alpgen+Herwig sample, used in the previous data to simulation comparison plots is used as a reference sample for the
unfolding procedure.
The systematic uncertainty associated with using a particular reference sample is estimated.
The propagation of uncertainties
through the unfolding method is described in Section~\ref{sec:ttjets_unfold_unc}.

The particle-level selection demands one electron or muon and no other lepton with the same pseudo-rapidity and transverse momentum
requirements as the reconstruction-level selection. The electron's four-momentum is summed with photons around it in a $\Delta R < 0.1$ region,
to simulate the effect of radiation emitted when interacting with the detector.
The sum of 
final state particles' transverse momentum around the lepton (excluding its own momentum and disregarding neutrinos) with
$p_T > 500\mev$ and $\Delta R < 0.3$ is required to be smaller than $2\gev$, to simulate the acceptance effect of the isolation cuts applied to the
leptons in the reconstruction-level selection~\footnote{For electrons, photons in the $\Delta R < 0.1$ region are excluded from the particle-level
isolation calculation, since they were used to built the electron's four-momentum, simulating its interaction with the detector.}.
The same overlap removal criteria are also demanded for the leptons, as in the reconstruction-level selection,
to simulate their acceptance.
The jets are built at the particle-level selection, using the Anti-$k_t$ algorithm with the same $R=0.4$ configuration applied to the simulation meta-data and the
same transverse momentum and pseudorapidity ranges are demanded.
The missing transverse energy at particle-level is calculated summing the simulation meta-data for particles that interact with the detector and taking their negative
transverse momentum. The same missing transverse energy and transverse mass requirements are applied in the simulation.
At least one of the particle-level jets is required to satisfy $\Delta R < 0.3$ between the jet axis and a $B$-hadron, to simulate the $b$-tagging criteria
demanded in the reconstruction-level selection.

The corrections in this method are expressed as:

\begin{eqnarray}
\Njetsp (i)&=& \mathcal{U}_{j \rightarrow i} [ \Njetsr (j) - \Njetsbkg (j) ] \nonumber \\
&=&\frac{1}{\ensuremath{f_{\mathrm{reco}}}(i)} \sum_{j=3}^{8} f(i, j)
(1 - \ensuremath{f_{\mathrm{np3}}}(j))(1 - \ensuremath{f^{'}_{\mathrm{fakes}}}(j) ) \times \nonumber \\
&& [ \Njetsr (j) - \Njetsbkg (j) ]
\label{eq:unfoldingB}
\end{eqnarray}
where $\Njetsr(j)$ represents the number of entries at reconstruction-level
jet multiplicity $j$ and $\Njetsp(i)$ represents the number of entries at
particle-level jet multiplicity $i$. Whenever the index is not explicitly
mentioned and the lower case variables $\njetsr$ and $\njetsp$ are used,
they will be taken to mean the jet multiplicity
values for a single event. The operator $\mathcal{U}_{j \rightarrow i}$ represents the unfolding process to be applied
in background subtracted input and it is defined by the equation above.

Starting with the reconstructed jet multiplicity spectrum
$\Njetsr (j)$, the background as estimated in
Section~\ref{sec:ttjets_background} is subtracted through the $\Njetsbkg (j)$
term.

The next steps are represented by the unfolding operator $\mathcal{U}_{j \rightarrow i}$.
In the steps that follow describing the unfolding operator,
the jet multiplicity requirement at the reconstruction-level selection or the particle-level
selection are not taken into account unless explicitly mentioned. The reason for this is that
this selection requirement is analysed independently. That means that, for what follows, ``reconstructed events'' will
be used to refer to events that pass the reconstruction-level selection, with no requirement on \njetsr, and
``particle-level events'' refers to events that pass the particle-level selection, with no requirement on \njetsp.
The abbreviation ``R'' is used for events that pass the reconstruction-level
selection, regardless of the \njetsr requirement and ``P'' for
events that pass the particle-level selection, regardless of the
\njetsp requirement.

The unfolding steps in the unfolding operator start (from right to left in Equation~\ref{eq:unfoldingB}) with
the removal of events that were reconstructed
but that fail the particle level cuts except the jet multiplicity
requirement.
The fraction of events reconstructed with
$\njetsr \geq 3$ that failed the particle level cuts is defined as
$f^{\prime}_{\mathrm{fakes}}$:

\begin{equation}
\displaystyle
f_{\textrm{fakes}}^{\prime} = \frac{\textrm{Number of events in ``R'' with } \njetsr \geq 3 \textrm{, but not in ``P''}}{\textrm{Number of events in ``R'' with } \njetsr \geq 3}.
\label{eq:ttjets_fpfakes}
\end{equation}
This factor estimates the fraction
of fake \ttbar events reconstructed in the signal sample used.
The value calculated for $1 - f^{\prime}_{\mathrm{fakes}}$ is shown in Figure~\ref{fig:MethodBffakesp}.
It is important to mention that the jet multiplicity is a consequence of the jet
requirements applied, that is, the acceptance cuts.
This correction factor includes acceptance effects in the reconstruction procedure related to the jets,
but not the particle-level effects. Note, as well, that Figure~\ref{fig:btag_ptrel} shows the $b$-tagging selection efficiency, which is
no more than 20\%, which has a major impact in the acceptance. The $f^{\prime}_{\mathrm{fakes}}$ factor also contains a contribution
from electron and muon misidentification and the jet vertex fraction selection requirement.

Since the particle-level jet multiplicity cut was disregarded in the previous
step, it must be taken into account separately. The $f_{\mathrm{np3}}$ factor is the
fraction of events that pass the reconstruction-level and particle-level
selections, but failed the requirement $\njetsp \geq 3$:

\begin{equation}
\displaystyle
f_{np3} = \frac{\textrm{Number of events in ``R'' (with } \njetsr \geq 3 \textrm{) and in ``P'', but fail }\njetsp \geq 3}{\textrm{Number of events in ``R'' (with } \njetsr \geq 3 \textrm{) and in ``P''}}.
\label{eq:ttjets_fnp3}
\end{equation}
The
multiplication by $1 - f_{\mathrm{np3}}$ removes events that migrated from
particle-level bins 0, 1 and 2 to reconstruction-level bins $\geq 3$.
The value calculated for $1 - f_{\mathrm{np3}}$ is shown in Figure~\ref{fig:MethodBfnp3}.

A migration correction from reconstruction-level to particle-level
is applied as a matrix multiplication.
Each element in the migration matrix, $f$,
is the conditional probability
that an event was at particle-level bin $j$,
given that it was reconstructed at bin $i$, that is:

\begin{equation}
\displaystyle
f(i, j) = \frac{\textrm{Number of events in ``R'' and ``P'' with }\njetsp = j \textrm{ and }\njetsr = i}{\textrm{Number of events in ``R'' with }\njetsr = i}.
\label{eq:ttjets_f}
\end{equation}
The factor $f$ is already an \emph{unsmearing} factor
which can be directly multiplied in Equation~\ref{eq:unfoldingB}.
The factor $f$ is calculated by counting the number of events in the $\njetsr$
and $\njetsp$ bins and normalising the matrix by the reconstruction-level bins,
so that $\sum_{j=3}^{8} f(i, j) = 1 \forall i \in [3,8]$.
The migration factor $f$ for the selection with a jet $p_T$ cut at 25 GeV
is shown
in Figure~\ref{fig:MethodBmigration}.

Finally, a correction is applied to include events that exist
at particle level, but that were lost
during the reconstruction procedure.
The fraction of events that exist at reconstruction level,
given that they can be found at
particle level, is given by $f_{\mathrm{reco}}$:

\begin{equation}
\displaystyle
f_{\mathrm{reco}} = \frac{\textrm{Number of events in ``R'' and in ``P''}}{\textrm{Number of events in ``P''}}.
\label{eq:ttjets_freco}
\end{equation}
The value calculated for $f_{\mathrm{reco}}$ is shown in Figure~\ref{fig:MethodBfreco}.
Note that this effect is corrected after the unsmearing performed by the $f$ matrix and after the out-of-acceptance
correction between reconstruction and particle-level performed by the $f^{\prime}_{\textrm{fakes}}$ and $f_{\textrm{np3}}$.
As a consequence, this final step only corrects for the acceptance difference for the events accepted by the particle-level selection.

The knowledge
of $\njetsr$ for an event before the event selection is not trivial, since
an overlap removal is done between jets and electrons, which removes
one or zero jets, changing the value of $\njetsr$ by one or zero. However,
it is not obvious which electron is to be selected (if any) and $\njetsr$
cannot be pre-calculated exactly.
This method uses the number of reconstructed jets, $\njetsr$,
only at the stages
in which the reconstruction-level selection was fulfilled, which avoids this
difficulty.

The effect of each of the corrections is shown in
Figure~\ref{fig:MethodBclosure} for the Alpgen top-antitop sample,
assuming perfect background subtraction. In this case, the unfolded result
matches perfectly the particle-level result as expected, because the same sample was used to derive the correction factors
and to apply them. The figure shows that the method is able to
recover the original particle-level result from a reconstruction-level
measurement.

The equivalent plots for the selection jet $p_T$ cut at 40, 60 and 80 GeV
are available in Section~\ref{subsec:methodBotherpt}.


\begin{figure}[htbp]
\centering
\subfloat{\includegraphics[width=0.49\textwidth]{figures/corrections/MethodB/FakesP_cut_AllEl.eps}}
\subfloat{\includegraphics[width=0.49\textwidth]{figures/corrections/MethodB/FakesP_cut_AllMu.eps}}
\caption{The $1 - f^{\prime}_{fakes}$ correction using the Alpgen \ttbar\ signal sample with a jet $p_T$ cut at 25 GeV.  The results for the electron (left) and muon (right) channels are shown.}
\label{fig:MethodBffakesp}
\end{figure}


\begin{figure}[htbp]
\centering
\subfloat{\includegraphics[width=0.49\textwidth]{figures/corrections/MethodB/Fnp3_cut_AllEl.eps}}
\subfloat{\includegraphics[width=0.49\textwidth]{figures/corrections/MethodB/Fnp3_cut_AllMu.eps}}
\caption{The $1 - f_{np3}$ correction using the Alpgen  \ttbar\ signal sample with a jet $p_T$ cut at 25 GeV.  The results for the electron (left) and muon (right) channels are shown.}
\label{fig:MethodBfnp3}
\end{figure}


\begin{figure}[htbp]
\centering
\subfloat{\includegraphics[width=0.49\textwidth]{figures/corrections/MethodB/fij_cut_AllEl.eps}}
\subfloat{\includegraphics[width=0.49\textwidth]{figures/corrections/MethodB/fij_cut_AllMu.eps}}
\caption{The migration matrix using the Alpgen  \ttbar\ signal sample with a selection using a jet $p_T$ cut at 25 GeV.  The results for the electron (left) and muon (right) channels are shown.}
\label{fig:MethodBmigration}
\end{figure}

\begin{figure}[htbp]
\centering
\subfloat{\includegraphics[width=0.49\textwidth]{figures/corrections/MethodB/fReco_cut_AllEl.eps}}
\subfloat{\includegraphics[width=0.49\textwidth]{figures/corrections/MethodB/fReco_cut_AllMu.eps}}
\caption{The $f_{reco}$ correction using the Alpgen  \ttbar\ signal sample with a jet $p_T$ cut at 25 GeV.  The results for the electron (left) and muon (right) channels are shown.}
\label{fig:MethodBfreco}
\end{figure}

\begin{sidewaysfigure}[htbp]
\centering
\subfloat{\includegraphics[width=0.49\textwidth]{figures/corrections/MethodB/PeterCorrections_cut_AllEl.eps}}
\subfloat{\includegraphics[width=0.49\textwidth]{figures/corrections/MethodB/PeterCorrections_cut_AllMu.eps}}
\caption{The closure test using the Alpgen  \ttbar\ signal sample for input and corrections with a jet $p_T$
  cut at 25 GeV for the selection.  The
  results for the electron (left) and muon (right) channels are shown.}
\label{fig:MethodBclosure}
\end{sidewaysfigure}

\section{Propagation of systematic uncertainties through the unfolding procedure}
\label{sec:ttjets_unfold_unc}

There are different sets of uncertainties that should be propagated through the method described previously.
The background is subtracted from data using a bin-by-bin subtraction of the data histogram and the reconstruction systematic variations of the backgrounds
are propagated to the background-subtracted data sample~\footnote{A positive systematic in the background would become a negative systematic variation in the new sample,
but the absolute value of the variations would not change.}. This results in a set of systematic uncertainties which need to be
propagated through the $\mathcal{U}_{j \rightarrow i}$ operator described previously.

The systematic uncertainty in the reconstruction of the \ttbar signal sample
used to estimate the unfolding correction factors should be considered separately, since the $\njetsr$ variable calculated in this sample
is not actually used in the unfolding procedure
except indirectly through the usage of the correction factors.
%This uncertainty has already been estimated for the $\njetsr$ variable,
%as it has been shown in the data to simulation comparison.
It is proposed
to calculate the fraction of the uncertainty in each $\Njetsr$ bin for the \ttbar sample used in deriving the correction factors and apply this fraction as a variation
to the background-subtracted data sample.
Note as well that this systematic variation is fully anti-correlated with the background reconstruction systematic variation
in the background-subtracted data,
since a positive variation in the background sample causes a negative variation in the background-subtracted sample. Therefore,  the background-subtracted data is multiplied by the positive fractional variation of the simulation \ttbar sample to get a negative systematic variation to be added in the
equivalent background systematic variation.

With the previous set of systematic uncertainties in the real data sample, a set of variations should be found in $I(j) = \Njetsr(j) - \Njetsbkg (j)$ from real data,
which incorporate the systematic effect of the reconstruction in the signal and background estimates.
It is not assumed that the propagation of uncertainties through the unfolding procedure $\Njetsp(i) = \mathcal{U}_{j \rightarrow i}[I(j)]$
leads to an unfolded uncertainty in $\Njetsp$ with the same distribution as the uncertainty in $I$.
The method implemented considers each source of systematic uncertainty $s$, at
each bin $j$ in the background-subtracted data sample $I$ separately. $\delta(s, j)$ is used to refer to the absolute systematic variation caused by $s$, in bin $j$, so that
the estimated sample with this variation is $I_s(s,j) = I(j) + \delta(s,j)$ for each bin $j$.
The systematic variation $\delta(s,j)$ represents a Gaussian standard deviation due to the systematic uncertainty source $s$ and it can
be calculated from the nominal background-subtracted data, $I(j)$, and the corresponding reconstruction-level systematic variation, $I_s(s,j)$, which is calculated
for each uncertainty source, as described in Section~\ref{sec:ttjets_syst}.
Pseudo-experiments are performed
to establish the effect of the unfolding procedure $\mathcal{U}_{j \rightarrow i}$ on the source $s$.

For each source $s$, a number $\mathcal{N}$ of pseudo-random samples of a Gaussian with mean zero and standard deviation one are taken and they are referred to as
$\alpha(s, m)$, for integers $m \in [1,\mathcal{N}]$. A pseudo-random systematic variation is defined as:

\begin{equation}
\displaystyle
I_{ps}(s, m, j) \triangleq I(j) + \alpha(s, m) \delta(s, j),
\label{eq:ttjets_unfold_pseudosyst}
\end{equation}
which has mean
$I(j)$ for each bin $j$ and standard deviation $\delta(s, j)$, as desired.
Each $I_{ps}(s, m, j)$ is unfolded into $\mathcal{U}_{j \rightarrow i}[I_{ps}(s, m, j)]$, which has a mean given by:

\begin{equation}
\displaystyle
\xi(s, i) \triangleq \frac{1}{\mathcal{N}} \sum_{m=1}^{\mathcal{N}} \mathcal{U}_{j \rightarrow i}[I_{ps}(s, m, j)].
\label{eq:ttjets_unfold_mean}
\end{equation}
The measure $\gamma(s, i)$
of the systematic effect of the source $s$ after the unfolding procedure is defined as the sample variance of $\mathcal{U}_{j \rightarrow i}[I_{ps}(s, m, j)]$:

\begin{equation}
\displaystyle
\gamma(s, i) \triangleq \frac{1}{\mathcal{N} - 1} \sqrt{\sum_{m = 1}^{\mathcal{N}} (\mathcal{U}_{j \rightarrow i}[I_{ps}(s, m, j)] - \xi(s, j))^2}.
\label{eq:ttjets_unfold_unc}
\end{equation}
This procedure gives us a $\gamma(s, i)$ for each bin $i$ and each systematic variation $s$, which is used as an estimate of this systematic variation after
the unfolding procedure. In this analysis, the number of pseudo-experiments was taken to be $\mathcal{N} = 1000$ due to computational limitations.

The procedure above is also implemented using the \ttbar simulation as an input.
The \ttbar simulation reconstruction-level histogram is represented as
$I_{\ttbar}(j)$, similarly to the $I(j)$ histogram for
background-subtracted data. All reconstruction-level systematics
in the \ttbar simulation can be propagated into an unfolded systematic
variation in the same way as it was described previously, resulting
in a measure of the systematic effect $s$ given by $\gamma_{\ttbar}(s, i)$
for the bin $i$ of the unfolded \ttbar $\Njetsptt(i) = \mathcal{U}_{j\rightarrow i}[I_{\ttbar}(j)]$.
The total effect of the reconstruction-level
systematic variation $s$ in the background-subtracted
and unfolded data is given by:

\begin{equation}
\displaystyle
\gamma_{\textrm{reco}}(s, i) \triangleq \gamma(s, i) - \Njetsp(i) \cdot \frac{\gamma_{\ttbar}(s, i)}{\Njetsptt(i)},
\label{eq:ttjets_totalsyst}
\end{equation}
while the nominal background-subtracted data is given by
$\Njetsp(i) = \mathcal{U}_{j\rightarrow i}[I(j)]$.
Note that the subtraction is used, since
the background and \ttbar systematic uncertainties are fully anti-correlated.

Another set of systematic uncertainties on the \ttbar modelling had a different treatment.
These uncertainties include the Parton Distribution Functions (PDFs), the initial state radiation and final state ratiation (ISR/FSR),
the parton shower modelling, the Monte Carlo generator systematics and the unfolding factors' statistical uncertainty~\footnote{The
samples used for each configuration are described in Section~\ref{sec:ttjets_background}. The statistical uncertainty in the unfolding factors
was taken into consideration, by applying the procedure described in this paragraph to a statistically independent sample to the \ttbar dataset used to
derive the unfolding correction factors, but which was generated in the same way. The MC@NLO sample was used to derive the Monte Carlo generator systematics.}.
The signal distribution at reconstruction-level
$I_{\textrm{model}}(c, j)$ for each different
configuration $c$ and bin $j$ is unfolded as $\mathcal{U}_{j \rightarrow i}[I_{\textrm{model}}(c, j)]$ and the systematic effect $\gamma_{\textrm{model}}(c, j)$ is defined as:

\begin{equation}
\displaystyle
\gamma_{\textrm{model}}(c, i) \triangleq |\mathcal{U}_{j \rightarrow i}[I_{\textrm{model}}(c, j)] - I_{\textrm{particle}}(c, i)|,
\end{equation}
where $I_{\textrm{particle}}(c, i)$ is the particle-level jet multiplicity value for configuration $c$, in bin $i$.
The total systematic uncertainty is calculated as:

\begin{eqnarray}
\displaystyle
\gamma_{\textrm{total}}(i)&\triangleq& \sqrt{ \sum_s \gamma_{\textrm{reco}}^2(s, i) + \sum_c \gamma_{\textrm{model}}^2(c, i) },
\label{eq:ttjets_totalsyst}
\end{eqnarray}
where it is implicit that for systematic uncertainties
that contain
asymmetric variations, the maximum (in absolute value) variation is used and
symmetrised and does not enter the sum.

The systematic uncertainties on the unfolded distributions are shown
in Tables~\ref{tab:unfSystElMethodB} and~\ref{tab:unfSystMuMethodB}.
The values shown are percentages of the unfolded data.

\input{tables/MethodBUnfoldingSystElAlpgen.tex}
\input{tables/MethodBUnfoldingSystMuAlpgen.tex}

\section{Results at particle level and discussion}
\label{sec:ttjets_unfolded}

The unfolded jet multiplicity distributions are shown in
Figures~\ref{fig:MethodBunfolded},~\ref{fig:MethodBunfolded40},~\ref{fig:MethodBunfolded60} and~\ref{fig:MethodBunfolded80}. These plots only
show the final number of entries after the correction implemented.
As a final step, the corrected number of entries was divided by the
integrated luminosity to estimate the fiducial cross section in
Figures~\ref{fig:xsMethodBunfolded},~\ref{fig:xsMethodBunfolded40},~\ref{fig:xsMethodBunfolded60} and~\ref{fig:xsMethodBunfolded80}.
The black line shows the unfolded data and the shaded band indicates the
propagated systematics. The green line shows the Alpgen \ttbar signal for
comparison. 
It can be seen that the unfolded data is compatible with the Alpgen \ttbar
signal.
% and with results provided by the main unfolding procedure, described in~\cite{ttjets_confnote}.
The plots in a logarithm scale for the Y axis are also shown in Figures~\ref{fig:MethodBunfoldedlog},~\ref{fig:MethodBunfolded40log},~\ref{fig:MethodBunfolded60log} and~\ref{fig:MethodBunfolded80log} for the corrected number
of events. The final fiducial cross section is shown in Figures~\ref{fig:xsMethodBunfoldedlog},~\ref{fig:xsMethodBunfolded40log},~\ref{fig:xsMethodBunfolded60log} and~\ref{fig:xsMethodBunfolded80log}.

\begin{sidewaysfigure}[htbp]
\centering
\subfloat{\includegraphics[width=0.49\textwidth]{figures/corrections/MethodB/PeterUnfoldedSyst_UnfoldFullSystCutEl.eps}}
\subfloat{\includegraphics[width=0.49\textwidth]{figures/corrections/MethodB/PeterUnfoldedSyst_UnfoldFullSystCutMu.eps}}
\caption{The unfolded data using the Alpgen  \ttbar\ signal sample for corrections.  The results for
  the electron (left) and muon (right) channels are shown. The systematic
  uncertainties from reconstruction and background estimation are
  included. The $p_T$ cut on the jets is 25~GeV.}
\label{fig:MethodBunfolded}
\end{sidewaysfigure}

\begin{sidewaysfigure}[htbp]
\centering
\subfloat{\includegraphics[width=0.49\textwidth]{figures/corrections/MethodB/PeterUnfoldedSyst_UnfoldFullSystCut40El.eps}}
\subfloat{\includegraphics[width=0.49\textwidth]{figures/corrections/MethodB/PeterUnfoldedSyst_UnfoldFullSystCut40Mu.eps}}
\caption{The unfolded data using the Alpgen  \ttbar\ signal sample for corrections.  The results for
  the electron (left) and muon (right) channels are shown. The systematic
  uncertainties from reconstruction and background estimation are
  included. The $p_T$ cut on the jets is 40~GeV. }
\label{fig:MethodBunfolded40}
\end{sidewaysfigure}

\begin{sidewaysfigure}[htbp]
\centering
\subfloat{\includegraphics[width=0.49\textwidth]{figures/corrections/MethodB/PeterUnfoldedSyst_UnfoldFullSystCut60El.eps}}
\subfloat{\includegraphics[width=0.49\textwidth]{figures/corrections/MethodB/PeterUnfoldedSyst_UnfoldFullSystCut60Mu.eps}}
\caption{The unfolded data using the Alpgen  \ttbar\ signal sample for corrections.  The results for
  the electron (left) and muon (right) channels are shown. The systematic
  uncertainties from reconstruction and background estimation are
  included. The $p_T$ cut on the jets is 60~GeV. }
\label{fig:MethodBunfolded60}
\end{sidewaysfigure}

\begin{sidewaysfigure}[htbp]
\centering
\subfloat{\includegraphics[width=0.49\textwidth]{figures/corrections/MethodB/PeterUnfoldedSyst_UnfoldFullSystCut80El.eps}}
\subfloat{\includegraphics[width=0.49\textwidth]{figures/corrections/MethodB/PeterUnfoldedSyst_UnfoldFullSystCut80Mu.eps}}
\caption{The unfolded data using the Alpgen  \ttbar\ signal sample for corrections.  The results for
  the electron (left) and muon (right) channels are shown. The systematic
  uncertainties from reconstruction and background estimation are
  included. The $p_T$ cut on the jets is 80~GeV. }
\label{fig:MethodBunfolded80}
\end{sidewaysfigure}

\clearpage

\begin{sidewaysfigure}[htbp]
\centering
\subfloat{\includegraphics[width=0.49\textwidth]{figures/corrections/MethodB/xs/PeterUnfoldedSyst_UnfoldFullSystCutEl.eps}}
\subfloat{\includegraphics[width=0.49\textwidth]{figures/corrections/MethodB/xs/PeterUnfoldedSyst_UnfoldFullSystCutMu.eps}}
\caption{The unfolded cross section using the Alpgen  \ttbar\ signal sample for corrections.  The results for
  the electron (left) and muon (right) channels are shown. The systematic
  uncertainties from reconstruction and background estimation are
  included. The $p_T$ cut on the jets is 25~GeV.}
\label{fig:xsMethodBunfolded}
\end{sidewaysfigure}

\begin{sidewaysfigure}[htbp]
\centering
\subfloat{\includegraphics[width=0.49\textwidth]{figures/corrections/MethodB/xs/PeterUnfoldedSyst_UnfoldFullSystCut40El.eps}}
\subfloat{\includegraphics[width=0.49\textwidth]{figures/corrections/MethodB/xs/PeterUnfoldedSyst_UnfoldFullSystCut40Mu.eps}}
\caption{The unfolded cross section using the Alpgen  \ttbar\ signal sample for corrections.  The results for
  the electron (left) and muon (right) channels are shown. The systematic
  uncertainties from reconstruction and background estimation are
  included. The $p_T$ cut on the jets is 40~GeV. }
\label{fig:xsMethodBunfolded40}
\end{sidewaysfigure}

\begin{sidewaysfigure}[htbp]
\centering
\subfloat{\includegraphics[width=0.49\textwidth]{figures/corrections/MethodB/xs/PeterUnfoldedSyst_UnfoldFullSystCut60El.eps}}
\subfloat{\includegraphics[width=0.49\textwidth]{figures/corrections/MethodB/xs/PeterUnfoldedSyst_UnfoldFullSystCut60Mu.eps}}
\caption{The unfolded cross section using the Alpgen  \ttbar\ signal sample for corrections.  The results for
  the electron (left) and muon (right) channels are shown. The systematic
  uncertainties from reconstruction and background estimation are
  included. The $p_T$ cut on the jets is 60~GeV. }
\label{fig:xsMethodBunfolded60}
\end{sidewaysfigure}

\begin{sidewaysfigure}[htbp]
\centering
\subfloat{\includegraphics[width=0.49\textwidth]{figures/corrections/MethodB/xs/PeterUnfoldedSyst_UnfoldFullSystCut80El.eps}}
\subfloat{\includegraphics[width=0.49\textwidth]{figures/corrections/MethodB/xs/PeterUnfoldedSyst_UnfoldFullSystCut80Mu.eps}}
\caption{The unfolded cross section using the Alpgen  \ttbar\ signal sample for corrections.  The results for
  the electron (left) and muon (right) channels are shown. The systematic
  uncertainties from reconstruction and background estimation are
  included. The $p_T$ cut on the jets is 80~GeV. }
\label{fig:xsMethodBunfolded80}
\end{sidewaysfigure}

\clearpage

It can be seen that the Alpgen+Pythia with the $\alpha_S$ variation upwards overestimates data at the high jet multiplicity bins, while
both Powheg and the Alpgen+Pythia with the downwards $\alpha_S$ variation agrees very well with data in all bins for the $p_T > 25\gev$ analysis.
The nominal Alpgen+Pythia described data within the systematic uncertainty, but not as well as the version with the downwards $\alpha_S$ variation.
Only \ttbar with one extra parton is simulated at the MC@NLO simulation,
which has a good estimate of the production cross section in the 3 and 4 jets bin. 
The MC@NLO simulation heavily underestimates data at large jet multiplicity, while Alpgen, which is a leading order generator, but includes up to 5 extra partons.

A complementary analysis to this one performed using the ATLAS detector is the ``jet gap fraction'' analysis~\cite{gapfraction}, which studies the
two-lepton final state of the \ttbar system and calculates the cross section of \ttbar production ($\sigma_{\ttbar}$) and the cross section disregarding
events with an extra jet produced with $p_T > Q_0$. The measured ratio is:

\begin{equation}
\displaystyle
f(Q_0) = \frac{\sigma_{\ttbar}(Q_0)}{\sigma_{\ttbar}}.
\label{eq:jgfq0}
\end{equation}
This analysis also measures the effect of extra radiation in \ttbar decays, for the dilepton final state and it has shown that the simulation prediction
has a large systematic effect, while the data uncertainty is smaller. It is interesting to compare the results of the analysis described in this document and
the results for the jet gap fraction analysis.
Figure~\ref{fig:jgf_Q0_08} shows the value of $f(Q_0)$ for the rapidity range $|y| < 0.8$ and it is clear that the jet gap fraction is over-estimated
in MC@NLO, which means that the fraction of events in MC@NLO with extra jets is smaller than data, in agreement with the results in this document.
The Alpgen+Herwig result in the jet gap fraction analysis also shows a better description of the extra radiation, although in this analysis the Alpgen+Pythia
and variations are not included.

\clearpage

\begin{sidewaysfigure}[htbp]
\centering
\subfloat{\includegraphics[width=0.49\textwidth]{figures/corrections/MethodB/log/PeterUnfoldedSyst_UnfoldFullSystCutEl.eps}}
\subfloat{\includegraphics[width=0.49\textwidth]{figures/corrections/MethodB/log/PeterUnfoldedSyst_UnfoldFullSystCutMu.eps}}
\caption{The unfolded data using the Alpgen \ttbar\ signal sample for corrections in logarithm
  scale for the Y axis.  The results for
  the electron (left) and muon (right) channels are shown. The systematic
  uncertainties from reconstruction and background estimation are
  included. The $p_T$ cut on the jets is 25~GeV.}
\label{fig:MethodBunfoldedlog}
\end{sidewaysfigure}

\begin{sidewaysfigure}[htbp]
\centering
\subfloat{\includegraphics[width=0.49\textwidth]{figures/corrections/MethodB/log/PeterUnfoldedSyst_UnfoldFullSystCut40El.eps}}
\subfloat{\includegraphics[width=0.49\textwidth]{figures/corrections/MethodB/log/PeterUnfoldedSyst_UnfoldFullSystCut40Mu.eps}}
\caption{The unfolded data using the Alpgen  \ttbar\ signal sample for corrections in logarithm
  scale for the Y axis.  The results for
  the electron (left) and muon (right) channels are shown. The systematic
  uncertainties from reconstruction and background estimation are
  included. The $p_T$ cut on the jets is 40~GeV. }
\label{fig:MethodBunfolded40log}
\end{sidewaysfigure}

\begin{sidewaysfigure}[htbp]
\centering
\subfloat{\includegraphics[width=0.49\textwidth]{figures/corrections/MethodB/log/PeterUnfoldedSyst_UnfoldFullSystCut60El.eps}}
\subfloat{\includegraphics[width=0.49\textwidth]{figures/corrections/MethodB/log/PeterUnfoldedSyst_UnfoldFullSystCut60Mu.eps}}
\caption{The unfolded data using the Alpgen  \ttbar\ signal sample for corrections in logarithm
  scale for the Y axis.  The results for
  the electron (left) and muon (right) channels are shown. The systematic
  uncertainties from reconstruction and background estimation are
  included. The $p_T$ cut on the jets is 60~GeV. }
\label{fig:MethodBunfolded60log}
\end{sidewaysfigure}

\begin{sidewaysfigure}[htbp]
\centering
\subfloat{\includegraphics[width=0.49\textwidth]{figures/corrections/MethodB/log/PeterUnfoldedSyst_UnfoldFullSystCut80El.eps}}
\subfloat{\includegraphics[width=0.49\textwidth]{figures/corrections/MethodB/log/PeterUnfoldedSyst_UnfoldFullSystCut80Mu.eps}}
\caption{The unfolded data using the Alpgen  \ttbar\ signal sample for corrections in logarithm
  scale for the Y axis.  The results for
  the electron (left) and muon (right) channels are shown. The systematic
  uncertainties from reconstruction and background estimation are
  included. The $p_T$ cut on the jets is 80~GeV. }
\label{fig:MethodBunfolded80log}
\end{sidewaysfigure}

% Cross sections
\clearpage

\begin{sidewaysfigure}[htbp]
\centering
\subfloat{\includegraphics[width=0.49\textwidth]{figures/corrections/MethodB/xs/log/PeterUnfoldedSyst_UnfoldFullSystCutEl.eps}}
\subfloat{\includegraphics[width=0.49\textwidth]{figures/corrections/MethodB/xs/log/PeterUnfoldedSyst_UnfoldFullSystCutMu.eps}}
\caption{The unfolded cross section using the Alpgen \ttbar\ signal sample for corrections in logarithm
  scale for the Y axis.  The results for
  the electron (left) and muon (right) channels are shown. The systematic
  uncertainties from reconstruction and background estimation are
  included. The $p_T$ cut on the jets is 25~GeV.}
\label{fig:xsMethodBunfoldedlog}
\end{sidewaysfigure}

\begin{sidewaysfigure}[htbp]
\centering
\subfloat{\includegraphics[width=0.49\textwidth]{figures/corrections/MethodB/xs/log/PeterUnfoldedSyst_UnfoldFullSystCut40El.eps}}
\subfloat{\includegraphics[width=0.49\textwidth]{figures/corrections/MethodB/xs/log/PeterUnfoldedSyst_UnfoldFullSystCut40Mu.eps}}
\caption{The unfolded cross section using the Alpgen  \ttbar\ signal sample for corrections in logarithm
  scale for the Y axis.  The results for
  the electron (left) and muon (right) channels are shown. The systematic
  uncertainties from reconstruction and background estimation are
  included. The $p_T$ cut on the jets is 40~GeV. }
\label{fig:xsMethodBunfolded40log}
\end{sidewaysfigure}

\begin{sidewaysfigure}[htbp]
\centering
\subfloat{\includegraphics[width=0.49\textwidth]{figures/corrections/MethodB/xs/log/PeterUnfoldedSyst_UnfoldFullSystCut60El.eps}}
\subfloat{\includegraphics[width=0.49\textwidth]{figures/corrections/MethodB/xs/log/PeterUnfoldedSyst_UnfoldFullSystCut60Mu.eps}}
\caption{The unfolded cross section using the Alpgen  \ttbar\ signal sample for corrections in logarithm
  scale for the Y axis.  The results for
  the electron (left) and muon (right) channels are shown. The systematic
  uncertainties from reconstruction and background estimation are
  included. The $p_T$ cut on the jets is 60~GeV. }
\label{fig:xsMethodBunfolded60log}
\end{sidewaysfigure}

\begin{sidewaysfigure}[htbp]
\centering
\subfloat{\includegraphics[width=0.49\textwidth]{figures/corrections/MethodB/xs/log/PeterUnfoldedSyst_UnfoldFullSystCut80El.eps}}
\subfloat{\includegraphics[width=0.49\textwidth]{figures/corrections/MethodB/xs/log/PeterUnfoldedSyst_UnfoldFullSystCut80Mu.eps}}
\caption{The unfolded cross section using the Alpgen  \ttbar\ signal sample for corrections in logarithm
  scale for the Y axis.  The results for
  the electron (left) and muon (right) channels are shown. The systematic
  uncertainties from reconstruction and background estimation are
  included. The $p_T$ cut on the jets is 80~GeV. }
\label{fig:xsMethodBunfolded80log}
\end{sidewaysfigure}

\clearpage

\begin{figure}[htbp]
\centering
\includegraphics[width=0.7\textwidth]{external/jgf_Q0_08.eps}
\caption{Jet gap fraction for $|y| < 0.8$, extracted from~\cite{gapfraction}.}
\label{fig:jgf_Q0_08}
\end{figure}

\clearpage

\section{Correction factors and consistency checks for selections with jet cuts at 40 GeV, 60 GeV and 80 GeV}
\label{subsec:methodBotherpt}

The closure tests for selection using jet $p_T$ cut at 40 GeV, 60 GeV and 80 GeV are shown in Figures~\ref{fig:MethodBclosure40},~\ref{fig:MethodBclosure60},~\ref{fig:MethodBclosure80}.
The $1 - f^{\prime}_{\mathrm{fakes}}$ factors for the selections using jet $p_T$ cut at 40 GeV, 60 GeV and 80 GeV are shown in the Figures~\ref{fig:MethodBffakesp40},~\ref{fig:MethodBffakesp60},~\ref{fig:MethodBffakesp80}.
The $1 - f_{\mathrm{np3}}$ factors for the selections using jet $p_T$ cut at 40 GeV, 60 GeV and 80 GeV are shown in the Figures~\ref{fig:MethodBfnp340},~\ref{fig:MethodBfnp360},~\ref{fig:MethodBfnp380}.
The $f$ migration correction factors for the selections using jet $p_T$ cut at 40 GeV, 60 GeV and 80 GeV are shown in the Figures~\ref{fig:MethodBmigration40},~\ref{fig:MethodBmigration60},~\ref{fig:MethodBmigration80}.
The $f_{reco}$ acceptance correction factors for the selections using jet $p_T$ cut at 40 GeV, 60 GeV and 80 GeV are shown in the Figures~\ref{fig:MethodBfreco40},~\ref{fig:MethodBfreco60},~\ref{fig:MethodBfreco80}.


\begin{sidewaysfigure}[htbp]
\centering
\subfloat{\includegraphics[width=0.49\textwidth]{figures/corrections/MethodB/PeterCorrections_cut40_AllEl.eps}}
\subfloat{\includegraphics[width=0.49\textwidth]{figures/corrections/MethodB/PeterCorrections_cut40_AllMu.eps}}
\caption{The closure test using the Alpgen  \ttbar\ signal sample for input and corrections with a jet $p_T$
  cut at 40 GeV for the selection.  The
  results for the electron (left) and muon (right) channels are shown.}
\label{fig:MethodBclosure40}
\end{sidewaysfigure}

\begin{figure}[htbp]
\centering
\subfloat{\includegraphics[width=0.49\textwidth]{figures/corrections/MethodB/FakesP_cut40_AllEl.eps}}
\subfloat{\includegraphics[width=0.49\textwidth]{figures/corrections/MethodB/FakesP_cut40_AllMu.eps}}
\caption{The $1 - f^{\prime}_{fakes}$ correction using the Alpgen \ttbar\ signal sample with a jet $p_T$ cut at 40 GeV.  The results for the electron (left) and muon (right) channels are shown.}
\label{fig:MethodBffakesp40}
\end{figure}


\begin{figure}[htbp]
\centering
\subfloat{\includegraphics[width=0.49\textwidth]{figures/corrections/MethodB/Fnp3_cut40_AllEl.eps}}
\subfloat{\includegraphics[width=0.49\textwidth]{figures/corrections/MethodB/Fnp3_cut40_AllMu.eps}}
\caption{The $1 - f_{np3}$ correction using the Alpgen \ttbar\ signal sample with a jet $p_T$ cut at 40 GeV.  The results for the electron (left) and muon (right) channels are shown.}
\label{fig:MethodBfnp340}
\end{figure}


\begin{figure}[htbp]
\centering
\subfloat{\includegraphics[width=0.49\textwidth]{figures/corrections/MethodB/fij_cut40_AllEl.eps}}
\subfloat{\includegraphics[width=0.49\textwidth]{figures/corrections/MethodB/fij_cut40_AllMu.eps}}
\caption{The migration matrix using the Alpgen \ttbar\ signal sample with a selection using a jet $p_T$ cut at 40 GeV.  The results for the electron (left) and muon (right) channels are shown.}
\label{fig:MethodBmigration40}
\end{figure}

\begin{figure}[htbp]
\centering
\subfloat{\includegraphics[width=0.49\textwidth]{figures/corrections/MethodB/fReco_cut40_AllEl.eps}}
\subfloat{\includegraphics[width=0.49\textwidth]{figures/corrections/MethodB/fReco_cut40_AllMu.eps}}
\caption{The $f_{reco}$ correction using the Alpgen \ttbar\ signal sample with a jet $p_T$ cut at 40 GeV.  The results for the electron (left) and muon (right) channels are shown.}
\label{fig:MethodBfreco40}
\end{figure}


\begin{sidewaysfigure}[htbp]
\centering
\subfloat{\includegraphics[width=0.49\textwidth]{figures/corrections/MethodB/PeterCorrections_cut60_AllEl.eps}}
\subfloat{\includegraphics[width=0.49\textwidth]{figures/corrections/MethodB/PeterCorrections_cut60_AllMu.eps}}
\caption{The closure test using the Alpgen  \ttbar\ signal sample for input and corrections with a jet $p_T$
  cut at 60 GeV for the selection.  The
  results for the electron (left) and muon (right) channels are shown.}
\label{fig:MethodBclosure60}
\end{sidewaysfigure}

\begin{figure}[htbp]
\centering
\subfloat{\includegraphics[width=0.49\textwidth]{figures/corrections/MethodB/FakesP_cut60_AllEl.eps}}
\subfloat{\includegraphics[width=0.49\textwidth]{figures/corrections/MethodB/FakesP_cut60_AllMu.eps}}
\caption{The $1 - f^{\prime}_{fakes}$ correction using the Alpgen \ttbar\ signal sample with a jet $p_T$ cut at 60 GeV.  The results for the electron (left) and muon (right) channels are shown.}
\label{fig:MethodBffakesp60}
\end{figure}


\begin{figure}[htbp]
\centering
\subfloat{\includegraphics[width=0.49\textwidth]{figures/corrections/MethodB/Fnp3_cut60_AllEl.eps}}
\subfloat{\includegraphics[width=0.49\textwidth]{figures/corrections/MethodB/Fnp3_cut60_AllMu.eps}}
\caption{The $1 - f_{np3}$ correction using the Alpgen \ttbar\ signal sample with a jet $p_T$ cut at 60 GeV.  The results for the electron (left) and muon (right) channels are shown.}
\label{fig:MethodBfnp360}
\end{figure}


\begin{figure}[htbp]
\centering
\subfloat{\includegraphics[width=0.49\textwidth]{figures/corrections/MethodB/fij_cut60_AllEl.eps}}
\subfloat{\includegraphics[width=0.49\textwidth]{figures/corrections/MethodB/fij_cut60_AllMu.eps}}
\caption{The migration matrix using the Alpgen \ttbar\ signal sample with a selection using a jet $p_T$ cut at 60 GeV.  The results for the electron (left) and muon (right) channels are shown.}
\label{fig:MethodBmigration60}
\end{figure}

\begin{figure}[htbp]
\centering
\subfloat{\includegraphics[width=0.49\textwidth]{figures/corrections/MethodB/fReco_cut60_AllEl.eps}}
\subfloat{\includegraphics[width=0.49\textwidth]{figures/corrections/MethodB/fReco_cut60_AllMu.eps}}
\caption{The $f_{reco}$ correction using the Alpgen \ttbar\ signal sample with a jet $p_T$ cut at 60 GeV.  The results for the electron (left) and muon (right) channels are shown.}
\label{fig:MethodBfreco60}
\end{figure}


\begin{sidewaysfigure}[htbp]
\centering
\subfloat{\includegraphics[width=0.49\textwidth]{figures/corrections/MethodB/PeterCorrections_cut80_AllEl.eps}}
\subfloat{\includegraphics[width=0.49\textwidth]{figures/corrections/MethodB/PeterCorrections_cut80_AllMu.eps}}
\caption{The closure test using the Alpgen  \ttbar\ signal sample for input and corrections with a jet $p_T$
  cut at 80 GeV for the selection.  The
  results for the electron (left) and muon (right) channels are shown.}
\label{fig:MethodBclosure80}
\end{sidewaysfigure}

\begin{figure}[htbp]
\centering
\subfloat{\includegraphics[width=0.49\textwidth]{figures/corrections/MethodB/FakesP_cut80_AllEl.eps}}
\subfloat{\includegraphics[width=0.49\textwidth]{figures/corrections/MethodB/FakesP_cut80_AllMu.eps}}
\caption{The $1 - f^{\prime}_{fakes}$ correction using the Alpgen \ttbar\ signal sample with a jet $p_T$ cut at 80 GeV.  The results for the electron (left) and muon (right) channels are shown.}
\label{fig:MethodBffakesp80}
\end{figure}


\begin{figure}[htbp]
\centering
\subfloat{\includegraphics[width=0.49\textwidth]{figures/corrections/MethodB/Fnp3_cut80_AllEl.eps}}
\subfloat{\includegraphics[width=0.49\textwidth]{figures/corrections/MethodB/Fnp3_cut80_AllMu.eps}}
\caption{The $1 - f_{np3}$ correction using the Alpgen  \ttbar\ signal sample with a jet $p_T$ cut at 80 GeV.  The results for the electron (left) and muon (right) channels are shown.}
\label{fig:MethodBfnp380}
\end{figure}


\begin{figure}[htbp]
\centering
\subfloat{\includegraphics[width=0.49\textwidth]{figures/corrections/MethodB/fij_cut80_AllEl.eps}}
\subfloat{\includegraphics[width=0.49\textwidth]{figures/corrections/MethodB/fij_cut80_AllMu.eps}}
\caption{The migration matrix using the Alpgen  \ttbar\ signal sample with a selection using a jet $p_T$ cut at 80 GeV.  The results for the electron (left) and muon (right) channels are shown.}
\label{fig:MethodBmigration80}
\end{figure}

\begin{figure}[htbp]
\centering
\subfloat{\includegraphics[width=0.49\textwidth]{figures/corrections/MethodB/fReco_cut80_AllEl.eps}}
\subfloat{\includegraphics[width=0.49\textwidth]{figures/corrections/MethodB/fReco_cut80_AllMu.eps}}
\caption{The $f_{reco}$ correction using the Alpgen  \ttbar\ signal sample with a jet $p_T$ cut at 80 GeV.  The results for the electron (left) and muon (right) channels are shown.}
\label{fig:MethodBfreco80}
\end{figure}

The systematic uncertainties for the unfolded data with the 40, 60 and 80 GeV
selections are given in Tables~\ref{tab:unfSyst40ElMethodB},~\ref{tab:unfSyst40MuMethodB},~\ref{tab:unfSyst60ElMethodB},~\ref{tab:unfSyst60MuMethodB},~\ref{tab:unfSyst80ElMethodB},~\ref{tab:unfSyst80MuMethodB}. All values are given as percentages of the unfolded data.

\clearpage

\input{tables/MethodBUnfoldingSyst40ElAlpgen.tex}
\input{tables/MethodBUnfoldingSyst40MuAlpgen.tex}

\input{tables/MethodBUnfoldingSyst60ElAlpgen.tex}
\input{tables/MethodBUnfoldingSyst60MuAlpgen.tex}

\input{tables/MethodBUnfoldingSyst80ElAlpgen.tex}
\input{tables/MethodBUnfoldingSyst80MuAlpgen.tex}

\clearpage

