
The \bjet trigger selects events
containing $b$-quarks that hadronise into jets.

% MOTIVATION
Selecting \bjet enriched samples is essential for many interesting physics analyses. Searches and measurements
involving
top quarks rely on a good identification of \bjets, since, as has been discussed in Chapter~\ref{chp:theory}, the top quark decays mainly into a
$W$-boson and a $b$-quark, due to the large value of the $|V_{tb}|$ matrix element~\cite{pdg2012}.
Besides top quark-related analyses, SM Higgs boson decays into bottom-antibottom quark pairs is an important signature, since this decay has a large branching
ratio~\cite{higgs_br} ($\sim 65\%$ for a Higgs mass of $120 \gev$, according to~\cite{higgs_br}).

However, due to the limited trigger bandwidth and storage facilities, one must be very selective on the choice of events to analyse. Trigger chains that select
\bjets are, therefore, an important element of the ATLAS Trigger system.

\section{\bjet trigger chains configuration}

The \bjet trigger signature is seeded by calorimeter-level seeds, in which a significant transverse energy deposition was measured. In the ATLAS Trigger, this
is implemented by the Level 1 jet trigger setup.
The Level 2 and Event Filter \bjet selection use the Inner Detector tracks that match the equivalent calorimeter region to apply a $b$-tagging
algorithm. It is important, therefore, to notice that, although the algorithm is seeded by an energy deposition in the calorimeters,
the $b$-tagging algorithm uses mainly tracks.

% EXPLAIN STRUCTURE
The part of the \bjet High-Level Trigger algorithm exclusively dedicated to $b$-tagging
is identical in the Level 2 and in the Event Filter, with the only difference being the usage of the Level 2 tracking algorithm or the Event Filter tracking algorithm,
with the calibration of the $b$-tagging algorithm adapted to the Level 2 or Event Filter tracking performance.
The basic structure involves a Feature Extraction algorithm that selects the trigger tracks and calculates a $b$-tagging weight (as described in the next section)
using the selected tracks; and
a Hypothesis testing algorithm that demands that the $b$-tagging weight is above a certain threshold. The Hypothesis algorithm is configured in three different ways for
three different types of trigger chains, which differ in their requirements for the $b$-tagging weight. The ``loose'' chains apply a requirement on
the $b$-tagging weight, so that the efficiency of \bjet selection is $\sim 40\%$; the ``medium'' chains have an efficiency of \bjet selection of $\sim 50\%$; while the
``tight'' chains, have a $\sim 60\%$ efficiency selection. There are many chains for each of these types, requiring different jet transverse momentum selections, and other
chains also make a requirement on lepton triggers.

A set of calibration chains are also available and they are frequently referred to as $\mu$-jet trigger chains. These chains use a simpler mechanism to select \bjet-enriched
samples: they rely on the semileptonic decays of the $B$-hadron into muons. They work by demanding that both a muon and a jet trigger chain passes, and that the muon and
jet directions are the same region of the detector. The latter requirement is implemented by demanding that $\Delta R(\mu, \textrm{jet}) < 0.4$. Some of the $\mu$-jet
trigger chains also restrict the difference in the $z$-coordinate of the jet and muon impact parameters.
To differentiate the $\mu$-jet calibration trigger chains from the ones used in physics analysis, the former are called ``calibration triggers'' in this chapter, while
the latter are called ``physics triggers''.

The \bjet physics trigger, chains use selected tracks which are required to have a minimum transverse momentum of $1\gev$,
a maximum $d_0$ of $1$ mm, and a $z_0$ relative to the primary vertex not exceeding $2$ mm.
The track fit quality must also be good and match the calorimeter-level seed.
The $z$ position of the primary vertex is calculated by histogramming all selected tracks and using a sliding window
algorithm to select the local maximum.

% b quark decay
The methods used to detect a \bjet rely on the fact that the $B$-hadron resulting from the \bjet hadronisation has a relatively long lifetime and it would
travel in the detector until it decays in a secondary vertex. Figure~\ref{fig:bjet_schem} shows a simplified
drawing representing the $B$-hadron decay, in which the $B$-hadron decays in a displaced secondary vertex. This, together with the large mass of the $B$-hadron,
generates tracks with high impact parameter (relative to the primary vertex), compared to light-jets.
The invariant mass
of the secondary vertex can also be used as part of the $b$-tagging procedure. The multiplicity of tracks coming from the $B$-hadron is also
used as part of the $b$-tagging procedure.
The next section describes the algorithm used to calculate the $b$-tagging weights in the \bjet physics trigger chains in more details.

\begin{figure}[tpbp]
\centering
\includegraphics[width=0.5\linewidth]{figures/bjet_schem.eps}
\caption{Simplified representation of the $B$-hadron decay in a secondary vertex. Extracted from~\cite{bjet_proc_pelle}.}
\label{fig:bjet_schem}
\end{figure}


\section{$b$-taggging algorithms}

Many $b$-tagging procedures can be used and there has been a change in the method implemented in the ATLAS \bjet trigger. The JetProb algorithm was implemented in
both Level 2 and Event Filter
for the 2011 data collection. The 2012 data collection used a combined tagger, which was calibrated based on the studies that are described in the next sections.
The JetProb procedure relies on selecting good quality tracks that match the calorimeter region and only uses the significance of their transverse impact parameter $d_0$,
defined by $S(d_0) = \pm \frac{d_0}{\sigma(d_0)}$. The sign of the impact parameter significances (both $S(z_0)$ and $S(d_0)$)
is distributed according to whether the tracks cross the jet cone axis in front of the primary vertex (positive) or behind it (negative).
The algorithm calculates the probability that each track comes from the primary vertex using a fitted resolution function
$\mathcal{R}(x)$, which is estimated as~\cite{bjet_proc_pelle}:

\begin{equation}
\displaystyle
\mathcal{P}_{\mathrm{track}} = \int_{-\infty}^{-|S(d_0)|} \mathcal{R}(x) dx.
\label{eq:jetprob}
\end{equation}
This probability is combined into a single probability that all tracks come from the primary vertex, which is expected to be very close to zero for
\bjets and uniformly distributed between zero and one for light-jets.
%The combined probability for the jet is defined as:
%
%\begin{equation}
%\displaystyle
%\mathcal{P}_{\mathrm{jet}} = \mathcal{P}_0 \sum_{i = 0}^{\textrm{N. of tracks} - 1} \frac{(- \ln \mathcal{P}_0)^i}{i !},
%\end{equation}
%where $\mathcal{P}_0 = \prod_{i=0}^{\textrm{N. of tracks} - 1} \mathcal{P}_{\mathrm{track } i}$.
%Consult~\cite{bjet_proc_pelle} for details.
This method is quite robust, since the $\mathcal{R}(x)$ distribution is
fitted from data using only light-jets (assuming the \bjet contribution is much smaller than the light-jets one), that is, it does not rely on Monte Carlo simulation.

An attempt has been made to switch to more efficient algorithms in 2012, but other algorithms rely on a good agreement between data and simulation
on many variables, including the transverse impact parameter significance, $S(d_0)$, the longitudinal impact parameter significance,
$S(z_0) = \frac{z_0 - z_{pv}}{\sigma(z_0)}$, and estimates of the kinematics of the tracks at the secondary vertex.

The impact parameter-related algorithms take advantage of the $b$-quark decay topology.
It is expected that the $b$-quark should hadronise into a $B$-hadron in a secondary vertex.
The jet origin would be displaced, due to the \bjet lifetime before decaying into the $B$-hadron and the tracks' perigee to the primary vertex, measured by their
impact parameters $z_0$ and $d_0$ and their uncertainties $\sigma(z_0)$ and $\sigma(d_0)$, would show this displacement relative to the jet cone axis
as an asymmetry in the signed impact parameter
significances. The IP3D algorithm builds a simulation-level likelihood of the $S(d_0)$ and $S(z_0)$ of tracks in \bjets and light-jets. A likelihood-ratio
is calculated in real data for all tracks combined and a cut is applied on it, to test the hypothesis that the jet indeed contains a \bjet.
Given the likelihoods $P_{b,d_0}(S(d_0))$ and $P_{b, z_0}(S(z_0))$ that represent the probability of a reconstructed \bjet's
track to have a transverse significance $S(d_0)$ and
a longitudinal significance $S(z_0)$, with similar definitions for the light-jet's tracks, $P_{l, d_0}(S(d_0))$ and $P_{l, z_0}(S(z_0))$, the IP3D weight $w_{\textrm{IP3D}}$
is defined as:

\begin{eqnarray}
\displaystyle
w^\prime_{\textrm{IP3D}} &\triangleq& \prod_{i=1}^{\textrm{N. of tracks}} \frac{P_{b,d_0}(S(d_0)_i)}{P_{l, d_0}(S(d_0)_i)} \frac{P_{b,z_0}(S(z_0)_i)}{P_{l, z_0}(S(z_0)_i)} \\
w_{\textrm{IP3D}} &\triangleq &\log_{10} \Big( w^\prime_{\textrm{IP3D}} \Big)
\label{eq:ip3d}
\end{eqnarray}

The secondary vertex algorithms reconstruct the invariant mass of the four-momentum sum of all tracks from the secondary vertex, the energy fraction of tracks coming from
the secondary vertex relative to all tracks in the jet, and the track multiplicity for tracks matched to the secondary vertex. These variables are also used in simulation
to calculate a likelihood function in signal and background, so that a likelihood-ratio can be estimated and a cut is applied in real data.
The algorithm used in the 2012 data taking in ATLAS used a combination of the IP3D and the secondary vertex algorithms described,
%by assuming that the likelihoods
%are uncorrelated and 
multiplying them into a single combined weight, on which a selection cut can be applied to verify whether the jet is a \bjet or not.
The combined weight is defined as:

\begin{eqnarray}
\displaystyle
w^\prime_{\textrm{COMB}} &\triangleq& w^\prime_{\textrm{E}} w^\prime_{\textrm{N}} w^\prime_{\textrm{M}} w^\prime_{\textrm{IP3D}} \nonumber \\
w_{\textrm{COMB}} &\triangleq& - \log_{10} \Big( 1 - \frac{w^\prime_{\textrm{COMB}}}{1+ w^\prime_{\textrm{COMB}}} \Big),
\label{eq:comb}
\end{eqnarray}
where $w^\prime_{\textrm{E}}$ is the likelihood ratio for the calculated secondary vertex energy, $w^\prime_{\textrm{N}}$ is the likelihood ratio for the
calculated number of tracks matched to the secondary vertex and $w^\prime_{\textrm{M}}$ is the likelihood ratio for the calculated invariant mass of
the secondary vertex.


The next section shows the
data to simulation agreement for some important variables, which are used to establish that there is indeed
a good agreement between data and simulation when switching to the new algorithm in the 2012 data taking.
The data to simulation comparison of the combined tagger used in the 2012 data taking is shown in the final section.

\section{Data to Monte Carlo simulation comparison of with $\sqrt{s} = 7$ TeV data}

A data to simulation comparison was prepared to check the reconstruction quality of the
Event Filter tracks.
The Event Filter significance of $d_0$, the tracks' multiplicity and the tracks' transverse momentum
is shown in Figures~\ref{fig:fromCDS_efSd0},~\ref{fig:fromCDS_eftrack_n_cut},~\ref{fig:fromCDS_eftrack_pt_cut}.
These plots show also the individual contribution of b-jets, c-jets and light-jets in simulation.
To determine the jet flavour in the simulated events, the Event Filter jets' $\eta$ and $\phi$ coordinates were used to match the trigger jets to the Anti-$k_t$ $R=0.4$
jets calculated by the Offline ATLAS software, which has access to the full detector information, with all jet energy corrections
applied. The standard Anti-$k_t$ ATLAS jets were matched to $B$-hadrons or $C$-hadrons, requiring a $\Delta R < 0.3$ between them,
to verify whether the jet is a \bjet, a $c$-jet, or a light-jet. In these figures, the simulation has been normalised to the total data yield.

\begin{figure}[H]
\centering
\includegraphics[width=0.7\linewidth]{bjet_datamc_2011/fromCDS_efSd0.eps}
\caption{The signed $S(d_0)$ for the selected tracks at the Event Filter, using 2011 ATLAS data.}
\label{fig:fromCDS_efSd0}
\end{figure}

\begin{figure}[H]
\centering
\includegraphics[width=0.7\linewidth]{bjet_datamc_2011/fromCDS_eftrack_n_cut.eps}
\caption{The track multiplicity for selected tracks at the Event Filter, using 2011 ATLAS data.}
\label{fig:fromCDS_eftrack_n_cut}
\end{figure}

\begin{figure}[H]
\centering
\includegraphics[width=0.7\linewidth]{bjet_datamc_2011/fromCDS_eftrack_pt_cut.eps}
\caption{The tracks' transverse momenta for selected tracks at the Event Filter, using 2011 ATLAS data.}
\label{fig:fromCDS_eftrack_pt_cut}
\end{figure}


%\input{bjet_nopublic.tex}

\section{Data to simulation comparison of the \bjet combined ``physics'' trigger using $\sqrt{s} = 8$ TeV, 2012 data}
\label{sec:datamc_physics}

A similar data to simulation comparison has been done with 2012 ATLAS data, but analysing the final $b$-tagging weight.
In this comparison, to avoid biasing the comparison in the tagger weight distribution, the trigger selection used applies no $b$-tagging cut
on the \bjet candidate in the ``physics trigger''~\footnote{In the ATLAS jargon, the trigger \texttt{EF\_b55\_NoCut\_j55\_a4tchad} was used.},
but only a cut on the transverse momentum of the jet candidate.
%The simulation dataset used were the di-jet QCD samples are mentioned in Appendix~\ref{sec:datasets}.

The transverse momentum cut is applied
on all \bjet physics selection chains, therefore this restriction presents no bias in the result, compared to other \bjet physics trigger chains.
Due to the high rate of events when the $b$-tagging cut is removed, this trigger is prescaled on data, which means that there will be a loss
of events. 
%Further cuts applied include \texttt{larError == 0 \&\& tileError != 2 \&\& ((coreFlags \& 0x40000) == 0)}, no jet with $p_T > 20$ GeV with \texttt{isBadLoose} flag.

%In the simulation datasets, the separation of events in different jet $p_T$ ranges works well at simulation-level, but after the reconstruction,
%migration between $p_T$ ranges cause large statistical fluctuations in the region when the transition between samples happen. To mitigate this effect,
%it was required that $\frac{p_T^{avg}}{p_T^{\textrm{truth,lead}}} < 2.0$, where $p_T^{avg}$ is the average transverse momentum
%between the two reconstructed leading jets and $p_T^{\textrm{truth,lead}}$ is the truth-level leading jet transverse momentum.

The combined tagger weight, on which the $b$-tagging cut is applied was plot, as well as the likelihood-ratio IP3D tagger, the number of tracks
at the Secondary Vertex, the energy fraction at the Secondary Vertex and the invariant mass of the Secondary Vertex, which are the relevant components for the combined tagger.
The 2012 ATLAS $\sqrt{s} = 8$ TeV data runs 213359 up to 213640 were used in this analysis, representing an integrated luminosity of $5.1 \inb$.
To understand the flavour fractions in these plots, the true flavour of the simulation-level jets can be investigated.
The HLT jets were geometrically matched to the Offline jets reconstructed with Anti-kt $R=0.4$ using a $\Delta R < 0.4$ criterion,
considering only $|\eta| < 2.5$ and $p_T > 40$ GeV offline jets.
The results are shown in Figures~\ref{fig:datamc_physics_flavour_comb},~\ref{fig:datamc_physics_flavour_ip3d},~\ref{fig:datamc_physics_flavour_sv_l2},~\ref{fig:datamc_physics_flavour_sv_ef},~\ref{fig:datamc_physics_flavour_msv}.
The Offline jets are matched to $b$- or $c$-hadrons in simulation to classify them as \bjets, $c$-jets or light-jets.
This allows one to see the simulation-level flavour fraction in the figures.
It can be seen that, for high values of the combined tagger weight, the \bjet component fraction is increased, as it is expected.
The modelling of the variables used in the combined tagger also seems to be consistently done in simulation, when compared to data.

\begin{figure}[H]
\subfloat[Level 2.]{
\includegraphics[width=0.48\linewidth]{bjet_datamc_2013/bjet/dataMCStack_l2comb_o_d3pd_C.eps}
}
\subfloat[Event Filter.]{
\includegraphics[width=0.48\linewidth]{bjet_datamc_2013/bjet/dataMCStack_efcomb_o_d3pd_C.eps}
}
\caption{Combined tagger weight for physics trigger using the impact parameter significance and the secondary vertex likelihood-based taggers, calculated from Level 2 and Event Filter tracks in low $p_T$ jets identified by the Level 1.}
\label{fig:datamc_physics_flavour_comb}
\end{figure}

\begin{figure}[H]
\subfloat[Level 2.]{
\includegraphics[width=0.48\linewidth]{bjet_datamc_2013/bjet/dataMCStack_l2ip3d_o_d3pd_C.eps}
}
\subfloat[Event Filter.]{
\includegraphics[width=0.48\linewidth]{bjet_datamc_2013/bjet/dataMCStack_efip3d_o_d3pd_C.eps}
}
\caption{IP3D tagger for physics trigger using the transverse and longitudinal impact parameter significances, calculated using Level 2 and Event Filter tracks in low $p_T$ jets identified by the Level 1.}
\label{fig:datamc_physics_flavour_ip3d}
\end{figure}

\begin{figure}[H]
\subfloat[Number of tracks from SV.]{
\includegraphics[width=0.48\linewidth]{bjet_datamc_2013/bjet/dataMCStack_l2nvtx_o_d3pd_C.eps}
}
\subfloat[Energy fraction in SV.]{
\includegraphics[width=0.48\linewidth]{bjet_datamc_2013/bjet/dataMCStack_l2evtx_o_d3pd_C.eps}
}
\caption{Data to simulation comparison for the physics trigger with flavour association for the Level 2 SV variables.}
\label{fig:datamc_physics_flavour_sv_l2}
\end{figure}

\begin{figure}[H]
\subfloat[Number of tracks from SV.]{
\includegraphics[width=0.48\linewidth]{bjet_datamc_2013/bjet/dataMCStack_efnvtx_o_d3pd_C.eps}
}
\subfloat[Energy fraction in SV.]{
\includegraphics[width=0.48\linewidth]{bjet_datamc_2013/bjet/dataMCStack_efevtx_o_d3pd_C.eps}
}
\caption{Data to simulation comparison for the physics trigger with flavour association for the Event Filter SV variables.}
\label{fig:datamc_physics_flavour_sv_ef}
\end{figure}

\begin{figure}[H]
\subfloat[Level 2.]{
\includegraphics[width=0.48\linewidth]{bjet_datamc_2013/bjet/dataMCStack_l2mvtx_o_d3pd_C.eps}
}
\subfloat[Event Filter.]{
\includegraphics[width=0.48\linewidth]{bjet_datamc_2013/bjet/dataMCStack_efmvtx_o_d3pd_C.eps}
}
\caption{Data to simulation comparison for the physics trigger with flavour association for the mass of the SV.}
\label{fig:datamc_physics_flavour_msv}
\end{figure}


\section{Data to simulation comparison in heavy flavour enriched sample with ATLAS 2012 $\sqrt{s} = 8$ TeV data}
\label{sec:datamc_calib}

A set of calibration triggers were also developed to get a sample of events enriched with heavy flavour content,
using an orthogonal selection to the main \bjet triggers. The idea relies on the semileptonic decay of the $b$-hadrons.
This allows the selection of a heavy flavour enriched sample,
by selecting jets which have muons nearby with $\Delta R < 0.4$.

For the analysis performed, a similar set of cuts was used, compared to the ``physics'' triggers analysis in the previous section,
but it was required that a muon and a jet trigger passed and that they are in the same $\eta \times \phi$ region of the detector~\footnote{In
the ATLAS jargon, the trigger
\texttt{EF\_mu4T\_j55\_a4tchad\_matched} was required to pass.}.
All other selection requirements, were done in accordance to the 
cuts described in Section~\ref{sec:datamc_physics}. The data runs used in this analysis range from 213359 up to 213640, representing
an integrated luminosity~\footnote{Note that the calibration trigger used here and the physics trigger without a $b$-tagging selection used in the
previous section are both prescaled by different amounts.} of $1.4 \ipb$.

Figures~\ref{fig:datamc_calib_flavour_comb} and~\ref{fig:datamc_calib_flavour_ip3d} show the Level 2 and Event Filter data to simulation
comparison using the calibration triggers for the combined and IP3D taggers respectively.
It can be noticed that the combined tagger weight has generally a larger fraction of events
at higher values, compared to the result in Section~\ref{sec:datamc_physics}, which means that this analysis is more sensitive to the
heavy flavour content on the data sample.
The track multiplicity and energy fraction data to simulation comparison are also shown in
Figures~\ref{fig:datamc_calib_flavour_sv_l2} and~\ref{fig:datamc_calib_flavour_sv_ef} for the Level 2 and Event Filter.
The estimated mass of the secondary vertex is shown in Figure~\ref{fig:datamc_calib_flavour_msv}.
These studies show that the modelling of the \bjet component in simulation is consistent with data and that the \bjet trigger
selection enhances the \bjet component for a cut on the combined tagger weight.

\begin{figure}[H]
\subfloat[Level 2.]{
\includegraphics[width=0.48\linewidth]{bjet_datamc_2013/mujet/dataMCStack_l2comb_o_d3pd_C.eps}
}
\subfloat[Event Filter.]{
\includegraphics[width=0.48\linewidth]{bjet_datamc_2013/mujet/dataMCStack_efcomb_o_d3pd_C.eps}
}
\caption{Combined tagger weight using the impact parameter significance and the secondary vertex likelihood-based taggers, calculated from Level 2 and Event Filter tracks in low $p_T$ jets identified by the Level 1.}
\label{fig:datamc_calib_flavour_comb}
\end{figure}

\begin{figure}[H]
\subfloat[Level 2.]{
\includegraphics[width=0.48\linewidth]{bjet_datamc_2013/mujet/dataMCStack_l2ip3d_o_d3pd_C.eps}
}
\subfloat[Event Filter.]{
\includegraphics[width=0.48\linewidth]{bjet_datamc_2013/mujet/dataMCStack_efip3d_o_d3pd_C.eps}
}
\caption{Data to simulation comparison for the calibration trigger with flavour association for the IP3D tagger.}
\label{fig:datamc_calib_flavour_ip3d}
\end{figure}


\begin{figure}[H]
\subfloat[Number of tracks from the SV.]{
\includegraphics[width=0.48\linewidth]{bjet_datamc_2013/mujet/dataMCStack_l2nvtx_o_d3pd_C.eps}
}
\subfloat[Energy fraction at the SV.]{
\includegraphics[width=0.48\linewidth]{bjet_datamc_2013/mujet/dataMCStack_l2evtx_o_d3pd_C.eps}
}
\caption{Data to simulation comparison for the calibration trigger with flavour association for the Level 2 SV variables.}
\label{fig:datamc_calib_flavour_sv_l2}
\end{figure}

\begin{figure}[H]
\subfloat[Number of tracks from the SV.]{
\includegraphics[width=0.48\linewidth]{bjet_datamc_2013/mujet/dataMCStack_efnvtx_o_d3pd_C.eps}
}
\subfloat[Energy fraction at the SV.]{
\includegraphics[width=0.48\linewidth]{bjet_datamc_2013/mujet/dataMCStack_efevtx_o_d3pd_C.eps}
}
\caption{Data to simulation comparison for the calibration trigger with flavour association for the Event Filter SV variables.}
\label{fig:datamc_calib_flavour_sv_ef}
\end{figure}

\begin{figure}[H]
\subfloat[Level 2.]{
\includegraphics[width=0.48\linewidth]{bjet_datamc_2013/mujet/dataMCStack_l2mvtx_o_d3pd_C.eps}
}
\subfloat[Event Filter.]{
\includegraphics[width=0.48\linewidth]{bjet_datamc_2013/mujet/dataMCStack_efmvtx_o_d3pd_C.eps}
}
\caption{Data to simulation comparison for the calibration trigger with flavour association for the mass of the SV.}
\label{fig:datamc_calib_flavour_msv}
\end{figure}

%\section{ROOT files generation and analysis framework}

%An analysis framework was developed by the student, to reproduce the trigger analysis
%without the full machinery needed for online data taking, allowing anlyses to be made on
%simple ROOT files, instead of complex ATLAS-specific files and software.
%The student is also the official contact person for any issue related to the ROOT file
%analysis framework, which has recently been updated to work on a more error proof procedure.
%In particular, the analysis framework used to be a copy of the online data taking software code,
%but now it uses the exact same C++ software code, with a few preprocessor definitions.

%The student also develops and mantains the software code necessary to actually dump the easy-to-use
%ROOT files mentioned in the previous paragraph using an ATLAS-specific paradigm.
%Another activity in which the student has been involved includes the coordination of the ROOT files
%automatic generation on the Grid, for all data taking done with the b-jet trigger.


\section{Summary}

The \bjet trigger allows one to enrich physics analysis samples with heavy flavour, which is very important for many analyses,
including top quark and Higgs boson related studies.
To perform reliable studies on the ATLAS \bjet trigger, it is important to test whether the simulation can describe real data well.
Furthermore, it is important to test whether the trigger selects heavy flavour events. This is essential for performance studies on the \bjet trigger.
In these studies, it is possible to verify that there is good agreement between data and simulation in the ATLAS data collected in 2011, which allowed the \bjet
trigger to be calibrated to use more complex likelihood-ratio-based taggers.

The data collected in ATLAS in 2012 uses a combined tagger, based on the
tracks' impact parameter and the secondary vertex observables. It has been shown that the combined tagger in the physics trigger enhances the \bjet component
for higher combined weight values. The agreement between data and simulation for the heavy flavour enhanced calibration triggers is also very good, showing that the \bjet
component is also correctly modelled.

