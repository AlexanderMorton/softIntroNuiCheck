
As mentioned in the introduction (Chapter~\ref{chp:introduction}), the top quark has an interesting position in the
Standard Model, due to its high mass compared to the other quarks and its small lifetime. It allows us to study a ``bare quark'' decay,
while other quarks hadronise. It has a strong coupling to the Higgs boson and the top-antitop production is a main background in many analyses,
such as the top-antitop-Higgs search. Furthermore, in many hypotheses Beyond the Standard Model it is expected that the top quark plays an important role with
particles which were not identified so far.

The ATLAS detector (Chapter~\ref{chp:atlas}) is an excellent environment to study the top quark physics
in detail. It is particularly important to detect \bjets in the detector, since the top quark decays 99\% of the times to a $W$-boson
and a $b$-quark. A performance study of the \bjet trigger in ATLAS was done in Chapter~\ref{chp:bjet}, showing that the detector is
ready to trigger on \bjets.

Aiming at a study of the Standard Model top quark production, a measurement of the top-antitop pair production cross section times the branching ratio in which
there is an electron or a muon in the final state was measured in data as a function of the jet multiplicity in Chapter~\ref{chp:ttbarjets}.
An unfolding procedure was devised to correct for the detector effects, with a systematic uncertainty estimate for the corrections.
The jet multiplicity in the final state
shows the effect of the QCD radiation in each bin. A comparison of Monte Carlo generators, including the matrix element generators and the parton shower simulation variations,
was done in the final unfolded result. It shows that MC@NLO and Alpgen+Pythia with the $\alpha_S$ increased variation describe the data very badly, but
Alpgen+Herwig, Powheg and Alpgen+Pythia with the $\alpha_S$ decreased variation describe the data better.

Although the Standard Model results can be measured directly using the top-antitop cross section analysis, another way of testing the Standard Model prediction
is to test the hypothesis that other models describe data better. A few models expect unverified particles to decay into top-antitop pairs and, in
Chapters~\ref{chp:ttbarres7} and~\ref{chp:ttbarres8},
they were tested with ATLAS data. The hypothesis that they are valid was excluded with 95\% Confidence Level for $\mtt \in [0.5\tev, 1.8\tev]$ for a Topcolor \zprime model
and $\mtt \in [0.5\tev, 2.07\tev]$ for the Kaluza-Klein gluons.

The analyses in the thesis show that, although there are still open points in the theory, the Standard Model describes the top-antitop production better than
the studied alternatives. It also has a good description of the top-antitop jet multiplicity for some Monte Carlo generators, while others need improvement.
