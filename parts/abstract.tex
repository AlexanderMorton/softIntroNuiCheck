This report is split into two self contained parts:

Resonance searches: \ref{chp:res} and Alignment studies: \ref{chp:GBL}.

The review concerned with the resonance searches briefly introduces the detector and the process of Monte Carlo simulations, before delving into the ongoing investigations.
   
The resolution of the merged $b\bar{b}$ track jets using a new generalised sequential recombination algorithm known as the variable R algorithm is explored. A particular focus is on the alignment of truth Monte Carlo B-hadrons with the reconstructed jet. Initial studies using standard model VH production show that the variable R jets better reconstruct the truth level B-hadrons direction compared to the anti-$k_t$ algorithm.  

The expected significance of signal over background for a 1/1.5 TeV resonance predicted by the Heavy Vector Triplet Model (HVT) is determined. This study comprises different discriminating variables being used in a window cut based multivariate analysis. Initial results show that a topological cuts based selection is almost optimum but more discriminating power may be added with some substructure variables.


Alignment studies will be performed on prototypes for the upgrade of the inner tracker of the ATLAS detector. Such prototypes will consist of complex pixel shapes, orientations and other varying environmental factors, which will need innovative methods to create a generic framework for common reconstruction. The EUTelescope framework is just that framework and has been updated to work with a plethora of different geometric and environmental setups. The initial results show the pattern recognition working in tandem with the track fitting and alignment. With results as expected for varying beam energies and magnetic fields. 
